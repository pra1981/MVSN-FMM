\documentclass[]{article}
\usepackage{lmodern}
\usepackage{amssymb,amsmath}
\usepackage{ifxetex,ifluatex}
\usepackage{fixltx2e} % provides \textsubscript
\ifnum 0\ifxetex 1\fi\ifluatex 1\fi=0 % if pdftex
  \usepackage[T1]{fontenc}
  \usepackage[utf8]{inputenc}
\else % if luatex or xelatex
  \ifxetex
    \usepackage{mathspec}
  \else
    \usepackage{fontspec}
  \fi
  \defaultfontfeatures{Ligatures=TeX,Scale=MatchLowercase}
\fi
% use upquote if available, for straight quotes in verbatim environments
\IfFileExists{upquote.sty}{\usepackage{upquote}}{}
% use microtype if available
\IfFileExists{microtype.sty}{%
\usepackage{microtype}
\UseMicrotypeSet[protrusion]{basicmath} % disable protrusion for tt fonts
}{}
\usepackage[margin=1in]{geometry}
\usepackage{hyperref}
\hypersetup{unicode=true,
            pdftitle={Simulation Tables},
            pdfauthor={Carter Allen},
            pdfborder={0 0 0},
            breaklinks=true}
\urlstyle{same}  % don't use monospace font for urls
\usepackage{graphicx,grffile}
\makeatletter
\def\maxwidth{\ifdim\Gin@nat@width>\linewidth\linewidth\else\Gin@nat@width\fi}
\def\maxheight{\ifdim\Gin@nat@height>\textheight\textheight\else\Gin@nat@height\fi}
\makeatother
% Scale images if necessary, so that they will not overflow the page
% margins by default, and it is still possible to overwrite the defaults
% using explicit options in \includegraphics[width, height, ...]{}
\setkeys{Gin}{width=\maxwidth,height=\maxheight,keepaspectratio}
\IfFileExists{parskip.sty}{%
\usepackage{parskip}
}{% else
\setlength{\parindent}{0pt}
\setlength{\parskip}{6pt plus 2pt minus 1pt}
}
\setlength{\emergencystretch}{3em}  % prevent overfull lines
\providecommand{\tightlist}{%
  \setlength{\itemsep}{0pt}\setlength{\parskip}{0pt}}
\setcounter{secnumdepth}{0}
% Redefines (sub)paragraphs to behave more like sections
\ifx\paragraph\undefined\else
\let\oldparagraph\paragraph
\renewcommand{\paragraph}[1]{\oldparagraph{#1}\mbox{}}
\fi
\ifx\subparagraph\undefined\else
\let\oldsubparagraph\subparagraph
\renewcommand{\subparagraph}[1]{\oldsubparagraph{#1}\mbox{}}
\fi

%%% Use protect on footnotes to avoid problems with footnotes in titles
\let\rmarkdownfootnote\footnote%
\def\footnote{\protect\rmarkdownfootnote}

%%% Change title format to be more compact
\usepackage{titling}

% Create subtitle command for use in maketitle
\providecommand{\subtitle}[1]{
  \posttitle{
    \begin{center}\large#1\end{center}
    }
}

\setlength{\droptitle}{-2em}

  \title{Simulation Tables}
    \pretitle{\vspace{\droptitle}\centering\huge}
  \posttitle{\par}
    \author{Carter Allen}
    \preauthor{\centering\large\emph}
  \postauthor{\par}
      \predate{\centering\large\emph}
  \postdate{\par}
    \date{4/21/2019}

\usepackage{booktabs}
\usepackage{longtable}
\usepackage{array}
\usepackage{multirow}
\usepackage{wrapfig}
\usepackage{float}
\usepackage{colortbl}
\usepackage{pdflscape}
\usepackage{tabu}
\usepackage{threeparttable}
\usepackage{threeparttablex}
\usepackage[normalem]{ulem}
\usepackage{makecell}
\usepackage{xcolor}

\begin{document}
\maketitle

\begin{table}[t]

\caption{\label{tab:unnamed-chunk-4}Model results for simulated data with n = 1500, k = 4, p = 1, h = 3, v = 1. 5000 iterations were run with a burn in of 1000. Missingness mechanism was MAR and P(miss) = 0}
\centering
\fontsize{8}{10}\selectfont
\begin{tabular}{llllllll}
\toprule
\multicolumn{2}{c}{ } & \multicolumn{2}{c}{Class 1} & \multicolumn{2}{c}{Class 2} & \multicolumn{2}{c}{Class 3} \\
\cmidrule(l{3pt}r{3pt}){3-4} \cmidrule(l{3pt}r{3pt}){5-6} \cmidrule(l{3pt}r{3pt}){7-8}
Model Component & Parameter & True & Est. (95\% CrI) & True & Est. (95\% CrI) & True & Est. (95\% CrI)\\
\midrule
\addlinespace[0.3em]
\multicolumn{8}{l}{\textbf{ }}\\
\hspace{1em}MVSN & $\beta_{0}$ & -3.21 & -3.34 (-3.8, -2.99) & -0.32 & -0.33 (-0.48, -0.14) & 3.35 & 3.33 (3.22, 3.44)\\
\hspace{1em}Regression & $\beta_{1}$ & -3.08 & -3.3 (-3.73, -2.87) & -0.75 & -0.72 (-0.87, -0.52) & 2.6 & 2.5 (2.39, 2.6)\\
\hspace{1em} & $\beta_{2}$ & -2.97 & -3.18 (-3.58, -2.76) & -0.45 & -0.44 (-0.58, -0.26) & 3.43 & 3.42 (3.31, 3.53)\\
\hspace{1em} & $\beta_{3}$ & -2.91 & -3.08 (-3.49, -2.68) & -0.66 & -0.68 (-0.83, -0.48) & 3.04 & 2.98 (2.87, 3.09)\\
\addlinespace[0.3em]
\multicolumn{8}{l}{\textbf{ }}\\
\hspace{1em} & $\sigma_{11}$ & 1 & 0.95 (0.84, 1.02) & 1 & 1 (0.89, 1.11) & 1 & 1.06 (0.97, 1.16)\\
\hspace{1em} & $\sigma_{12}$ & 0.74 & 0.7 (0.59, 0.76) & 0.68 & 0.68 (0.59, 0.78) & -0.45 & -0.41 (-0.47, -0.36)\\
\hspace{1em} & $\sigma_{13}$ & 0.74 & 0.69 (0.58, 0.75) & -0.16 & -0.13 (-0.2, -0.06) & 0.82 & 0.88 (0.79, 0.97)\\
\hspace{1em} & $\sigma_{14}$ & 0.98 & 0.93 (0.81, 0.99) & 0.64 & 0.65 (0.56, 0.75) & 0.7 & 0.75 (0.67, 0.83)\\
\hspace{1em} & $\sigma_{22}$ & 1 & 0.94 (0.82, 1.01) & 1 & 1.03 (0.93, 1.13) & 1 & 1.07 (0.99, 1.16)\\
\hspace{1em} & $\sigma_{23}$ & 0.83 & 0.79 (0.67, 0.85) & -0.43 & -0.4 (-0.46, -0.34) & -0.66 & -0.62 (-0.68, -0.57)\\
\hspace{1em} & $\sigma_{24}$ & 0.81 & 0.77 (0.66, 0.83) & 0.63 & 0.67 (0.58, 0.77) & 0.01 & 0.07 (0.01, 0.13)\\
\hspace{1em} & $\sigma_{33}$ & 1 & 0.96 (0.84, 1.03) & 1 & 1 (0.91, 1.09) & 1 & 1.05 (0.96, 1.15)\\
\hspace{1em} & $\sigma_{34}$ & 0.85 & 0.81 (0.69, 0.87) & 0.15 & 0.15 (0.08, 0.23) & 0.59 & 0.64 (0.56, 0.72)\\
\hspace{1em} & $\sigma_{44}$ & 1 & 0.95 (0.83, 1.01) & 1 & 1.02 (0.92, 1.13) & 1 & 1.06 (0.97, 1.15)\\
\addlinespace[0.3em]
\multicolumn{8}{l}{\textbf{ }}\\
\hspace{1em} & $\psi_{1}$ & -0.33 & -0.17 (-0.62, 0.39) & 0.67 & 0.7 (0.46, 0.89) & -1 & -1.01 (-1.13, -0.87)\\
\hspace{1em} & $\psi_{2}$ & -0.33 & -0.08 (-0.61, 0.44) & 0.67 & 0.63 (0.38, 0.82) & -1 & -0.88 (-1.01, -0.75)\\
\hspace{1em} & $\psi_{3}$ & -0.33 & -0.08 (-0.61, 0.39) & 0.67 & 0.64 (0.43, 0.82) & -1 & -1.01 (-1.14, -0.88)\\
\hspace{1em} & $\psi_{4}$ & -0.33 & -0.13 (-0.63, 0.37) & 0.67 & 0.7 (0.45, 0.89) & -1 & -0.94 (-1.07, -0.81)\\
\addlinespace[0.3em]
\multicolumn{8}{l}{\textbf{ }}\\
\hspace{1em}Multinom. & $\delta_{11}$ & 0.9 & 0.88 (0.81, 0.95) & 0.9 & 0.88 (0.81, 0.95) & 0.9 & 0.88 (0.81, 0.95)\\
\hspace{1em} & $\delta_{12}$ & 0.23 & 0.22 (0.14, 0.3) & 0.23 & 0.22 (0.14, 0.3) & 0.23 & 0.22 (0.14, 0.3)\\
\addlinespace[0.3em]
\multicolumn{8}{l}{\textbf{ }}\\
\hspace{1em}Clustering & $\pi_l$ & 0.28 & 0.28 (0.27, 0.28) & 0.42 & 0.43 (0.42, 0.43) & 0.3 & 0.3 (0.3, 0.3)\\
\bottomrule
\end{tabular}
\end{table}


\end{document}
