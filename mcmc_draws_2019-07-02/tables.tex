\documentclass[]{article}
\usepackage{lmodern}
\usepackage{amssymb,amsmath}
\usepackage{ifxetex,ifluatex}
\usepackage{fixltx2e} % provides \textsubscript
\ifnum 0\ifxetex 1\fi\ifluatex 1\fi=0 % if pdftex
  \usepackage[T1]{fontenc}
  \usepackage[utf8]{inputenc}
\else % if luatex or xelatex
  \ifxetex
    \usepackage{mathspec}
  \else
    \usepackage{fontspec}
  \fi
  \defaultfontfeatures{Ligatures=TeX,Scale=MatchLowercase}
\fi
% use upquote if available, for straight quotes in verbatim environments
\IfFileExists{upquote.sty}{\usepackage{upquote}}{}
% use microtype if available
\IfFileExists{microtype.sty}{%
\usepackage{microtype}
\UseMicrotypeSet[protrusion]{basicmath} % disable protrusion for tt fonts
}{}
\usepackage[margin=1in]{geometry}
\usepackage{hyperref}
\hypersetup{unicode=true,
            pdftitle={Simulation Tables},
            pdfauthor={Carter Allen},
            pdfborder={0 0 0},
            breaklinks=true}
\urlstyle{same}  % don't use monospace font for urls
\usepackage{graphicx,grffile}
\makeatletter
\def\maxwidth{\ifdim\Gin@nat@width>\linewidth\linewidth\else\Gin@nat@width\fi}
\def\maxheight{\ifdim\Gin@nat@height>\textheight\textheight\else\Gin@nat@height\fi}
\makeatother
% Scale images if necessary, so that they will not overflow the page
% margins by default, and it is still possible to overwrite the defaults
% using explicit options in \includegraphics[width, height, ...]{}
\setkeys{Gin}{width=\maxwidth,height=\maxheight,keepaspectratio}
\IfFileExists{parskip.sty}{%
\usepackage{parskip}
}{% else
\setlength{\parindent}{0pt}
\setlength{\parskip}{6pt plus 2pt minus 1pt}
}
\setlength{\emergencystretch}{3em}  % prevent overfull lines
\providecommand{\tightlist}{%
  \setlength{\itemsep}{0pt}\setlength{\parskip}{0pt}}
\setcounter{secnumdepth}{0}
% Redefines (sub)paragraphs to behave more like sections
\ifx\paragraph\undefined\else
\let\oldparagraph\paragraph
\renewcommand{\paragraph}[1]{\oldparagraph{#1}\mbox{}}
\fi
\ifx\subparagraph\undefined\else
\let\oldsubparagraph\subparagraph
\renewcommand{\subparagraph}[1]{\oldsubparagraph{#1}\mbox{}}
\fi

%%% Use protect on footnotes to avoid problems with footnotes in titles
\let\rmarkdownfootnote\footnote%
\def\footnote{\protect\rmarkdownfootnote}

%%% Change title format to be more compact
\usepackage{titling}

% Create subtitle command for use in maketitle
\providecommand{\subtitle}[1]{
  \posttitle{
    \begin{center}\large#1\end{center}
    }
}

\setlength{\droptitle}{-2em}

  \title{Simulation Tables}
    \pretitle{\vspace{\droptitle}\centering\huge}
  \posttitle{\par}
    \author{Carter Allen}
    \preauthor{\centering\large\emph}
  \postauthor{\par}
    \date{}
    \predate{}\postdate{}
  
\usepackage{booktabs}
\usepackage{longtable}
\usepackage{array}
\usepackage{multirow}
\usepackage{wrapfig}
\usepackage{float}
\usepackage{colortbl}
\usepackage{pdflscape}
\usepackage{tabu}
\usepackage{threeparttable}
\usepackage{threeparttablex}
\usepackage[normalem]{ulem}
\usepackage{makecell}
\usepackage{xcolor}

\begin{document}
\maketitle

\begin{table}[t]

\caption{\label{tab:unnamed-chunk-4}Model results for simulated data with n = 1000, k = 4, p = 2, h = 3, r = 2. 1000 iterations were run with a burn in of 250. Missingness mechanism was MAR and P(miss) = 0}
\centering
\fontsize{8}{10}\selectfont
\begin{tabular}{llllllll}
\toprule
\multicolumn{2}{c}{ } & \multicolumn{2}{c}{Class 1} & \multicolumn{2}{c}{Class 2} & \multicolumn{2}{c}{Class 3} \\
\cmidrule(l{3pt}r{3pt}){3-4} \cmidrule(l{3pt}r{3pt}){5-6} \cmidrule(l{3pt}r{3pt}){7-8}
Model Component & Parameter & True & Est. (95\% CrI) & True & Est. (95\% CrI) & True & Est. (95\% CrI)\\
\midrule
\addlinespace[0.3em]
\multicolumn{8}{l}{\textbf{ }}\\
\hspace{1em}MVSN & $\beta_{11}$ & -3.07 & -2.7 (-3.27, -2.34) & 0.42 & 0.9 (-0.06, 1.7) & 2.46 & 2.23 (1.75, 2.56)\\
\hspace{1em}Regression & $\beta_{21}$ & -2.04 & -1.99 (-2.17, -1.82) & -0.31 & -0.26 (-0.44, -0.02) & 3.26 & 3.27 (3.16, 3.38)\\
\hspace{1em} & $\beta_{31}$ & -3.03 & -3.14 (-3.55, -2.77) & 0.34 & 0.49 (-0.41, 1.35) & 2.93 & 2.77 (2.05, 3.09)\\
\hspace{1em} & $\beta_{41}$ & -3.26 & -3.24 (-3.4, -3.07) & -0.63 & -0.59 (-0.77, -0.38) & 2.53 & 2.54 (2.43, 2.65)\\
\hspace{1em} & $\beta_{12}$ & -3.12 & -3.48 (-3.88, -2.82) & 0.09 & 0.51 (-0.32, 1.27) & 2.67 & 2.22 (1.4, 2.57)\\
\hspace{1em} & $\beta_{22}$ & -2.61 & -2.62 (-2.77, -2.48) & -0.37 & -0.35 (-0.52, -0.15) & 2.1 & 2.07 (1.96, 2.18)\\
\hspace{1em} & $\beta_{32}$ & -2.84 & -2.8 (-3.4, -2.36) & -0.06 & 0.24 (-0.71, 1.12) & 1.89 & 1.57 (1.07, 1.87)\\
\hspace{1em} & $\beta_{42}$ & -2.8 & -2.62 (-2.79, -2.48) & 0.09 & 0.14 (-0.06, 0.34) & 3.38 & 3.35 (3.24, 3.45)\\
\addlinespace[0.3em]
\multicolumn{8}{l}{\textbf{ }}\\
\hspace{1em} & $\sigma_{11}$ & 1 & 1.13 (0.79, 1.63) & 1 & 1.13 (0.63, 4.43) & 1 & 1.27 (0.88, 1.76)\\
\hspace{1em} & $\sigma_{12}$ & 0.5 & 0.73 (0.46, 1.09) & 0.5 & 0.58 (0.21, 3.5) & 0.5 & 0.8 (0.48, 1.42)\\
\hspace{1em} & $\sigma_{13}$ & 0.25 & 0.47 (0.21, 0.71) & 0.25 & 0.29 (-0.01, 2.68) & 0.25 & 0.59 (0.32, 1.09)\\
\hspace{1em} & $\sigma_{14}$ & 0.12 & 0.21 (-0.04, 0.54) & 0.12 & 0.06 (-0.2, 2.72) & 0.12 & 0.32 (0.08, 0.76)\\
\hspace{1em} & $\sigma_{22}$ & 1 & 1.3 (1.03, 1.65) & 1 & 0.98 (0.53, 3.36) & 1 & 1.21 (0.91, 1.88)\\
\hspace{1em} & $\sigma_{23}$ & 0.5 & 0.73 (0.52, 1) & 0.5 & 0.58 (0.26, 2.72) & 0.5 & 0.85 (0.57, 1.36)\\
\hspace{1em} & $\sigma_{24}$ & 0.25 & 0.52 (0.31, 0.82) & 0.25 & 0.16 (-0.17, 2.34) & 0.25 & 0.5 (0.25, 1.02)\\
\hspace{1em} & $\sigma_{33}$ & 1 & 1.03 (0.78, 1.31) & 1 & 0.87 (0.5, 2.88) & 1 & 1.43 (1.12, 1.81)\\
\hspace{1em} & $\sigma_{34}$ & 0.5 & 0.68 (0.48, 0.92) & 0.5 & 0.36 (0.04, 2.38) & 0.5 & 0.84 (0.61, 1.29)\\
\hspace{1em} & $\sigma_{44}$ & 1 & 1.1 (0.75, 1.45) & 1 & 0.83 (0.45, 3.03) & 1 & 1.21 (0.93, 1.63)\\
\addlinespace[0.3em]
\multicolumn{8}{l}{\textbf{ }}\\
\hspace{1em} & $\psi_{1}$ & -0.67 & -0.92 (-1.34, -0.19) & 0.33 & -0.14 (-1.17, 0.97) & -1.33 & -1.19 (-1.58, -0.62)\\
\hspace{1em} & $\psi_{2}$ & -0.67 & -0.39 (-0.81, 0.09) & 0.33 & 0.14 (-0.93, 1.1) & -1.33 & -1.21 (-1.57, -0.38)\\
\hspace{1em} & $\psi_{3}$ & -0.67 & 0 (-0.83, 0.45) & 0.33 & -0.21 (-1.09, 0.75) & -1.33 & -0.87 (-1.33, 0.11)\\
\hspace{1em} & $\psi_{4}$ & -0.67 & -0.68 (-1.18, 0.09) & 0.33 & -0.08 (-1.19, 1.03) & -1.33 & -1.07 (-1.39, -0.5)\\
\addlinespace[0.3em]
\multicolumn{8}{l}{\textbf{ }}\\
\hspace{1em}Multinom. & $\delta_{11}$ & -0.5 & -0.29 (-0.54, -0.05) & -0.5 & -0.29 (-0.54, -0.05) & -0.5 & -0.29 (-0.54, -0.05)\\
\hspace{1em} & $\delta_{12}$ & 0.33 & 0.12 (-0.24, 0.48) & 0.33 & 0.12 (-0.24, 0.48) & 0.33 & 0.12 (-0.24, 0.48)\\
\hspace{1em} & $\delta_{21}$ & 0.36 & 0.33 (0.12, 0.53) & 0.36 & 0.33 (0.12, 0.53) & 0.36 & 0.33 (0.12, 0.53)\\
\hspace{1em} & $\delta_{22}$ & 0.96 & 0.98 (0.69, 1.28) & 0.96 & 0.98 (0.69, 1.28) & 0.96 & 0.98 (0.69, 1.28)\\
\addlinespace[0.3em]
\multicolumn{8}{l}{\textbf{ }}\\
\hspace{1em}Clustering & $\pi_l$ & 0.25 & 0.25 (0.22, 0.27) & 0.19 & 0.18 (0.15, 0.21) & 0.56 & 0.57 (0.55, 0.59)\\
\bottomrule
\end{tabular}
\end{table}


\end{document}
