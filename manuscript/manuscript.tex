%  ALWAYS USE THE referee OPTION WITH PAPERS SUBMITTED TO BIOMETRICS!!!
%  You can see what your paper would look like typeset by removing
%  the referee option.  Because the typeset version will be in two
%  columns, however, some of your equations may be too long. DO NOT
%  use the \longequation option discussed in the user guide!!!  This option
%  is reserved ONLY for equations that are impossible to split across 
%  multiple lines; e.g., a very wide matrix.  Instead, type your equations 
%  so that they stay in one column and are split across several lines, 
%  as are almost all equations in the journal.  Use a recent version of the
%  journal as a guide. 
%  
\documentclass[useAMS,referee]{biom}
\usepackage{amsmath}
\usepackage{graphicx}
\usepackage{blindtext}
\usepackage{booktabs}

\def\bSig\mathbf{\Sigma}
\newcommand{\VS}{V\&S}
\newcommand{\tr}{\mbox{tr}}

%  Here, place your title and author information.  Note that in 
%  use of the \author command, you create your own footnotes.  Follow
%  the examples below in creating your author and affiliation information.
%  Also consult a recent issue of the journal for examples of formatting.

\title[]{A multivariate skew-normal finite mixture model for analysis of infant development trajectories}

%  Here are examples of different configurations of author/affiliation
%  displays.  According to the Biometrics style, in some instances,
%  the convention is to have superscript *, **, etc footnotes to indicate 
%  which of multiple email addresses belong to which author.  In this case,
%  use the \email{ } command to produce the emails in the display.

%  In other cases, such as a single author or two authors from 
%  different institutions, there should be no footnoting.  Here, use
%  the \emailx{ } command instead. 

%  The examples below corrspond to almost every possible configuration
%  of authors and may be used as a guide.  For other configurations, consult
%  a recent issue of the the journal.

%  Single author -- USE \emailx{ } here so that no asterisk footnoting
%  for the email address will be produced.

%\author{John Author\emailx{email@address.edu} \\
%Department of Statistics, University of Warwick, Coventry CV4 7AL, U.K.}

%  Two authors from the same institution, with both emails -- use
%  \email{ } here to produce the asterisk footnoting for each email address

%\author{John Author$^{*}$\email{author@address.edu} and
%Kathy Authoress$^{**}$\email{email2@address.edu} \\
%Department of Statistics, University of Warwick, Coventry CV4 7AL, U.K.}

%  Exactly two authors from different institutions, with both emails  
%  USE \emailx{ } here so that no asterisk footnoting for the email address
%  is produced.

\author
{Carter Allen\emailx{allecart@musc.edu} \\
Department of Public Health Sciences, Medical University of South Carolina, Charleston, SC, U.S.A
\and
Brian Neelon, PhD \\
Department of Public Health Sciences, Medical University of South Carolina, Charleston, SC, U.S.A
\and
Sara Benjamin-Neelon, PhD, MPH, RD \\
Department of Health, Behavior and Society, Johns Hopkins University, Baltimore, MD, U.S.S}

%  Three or more authors from same institution with all emails displayed
%  and footnoted using asterisks -- use \email{ } 

%\author{John Author$^*$\email{author@address.edu}, 
%Jane Author$^{**}$\email{jane@address.edu}, and 
%Dick Author$^{***}$\email{dick@address.edu} \\
%Department of Statistics, University of Warwick, Coventry CV4 7AL, U.K}

%  Three or more authors from same institution with one corresponding email
%  displayed

%\author{John Author$^*$\email{author@address.edu}, 
%Jane Author, and Dick Author \\
%Department of Statistics, University of Warwick, Coventry CV4 7AL, U.K}

%  Three or more authors, with at least two different institutions,
%  more than one email displayed 

%\author{John Author$^{1,*}$\email{author@address.edu}, 
%Kathy Author$^{2,**}$\email{anotherauthor@address.edu}, and 
%Wilma Flinstone$^{3,***}$\email{wilma@bedrock.edu} \\
%$^{1}$Department of Statistics, University of Warwick, Coventry CV4 7AL, U.K \\
%$^{2}$Department of Biostatistics, University of North Carolina at 
%Chapel Hill, Chapel Hill, North Carolina, U.S.A. \\
%$^{3}$Department of Geology, University of Bedrock, Bedrock, Kansas, U.S.A.}

%  Three or more authors with at least two different institutions and only
%  one email displayed

%\author{John Author$^{1,*}$\email{author@address.edu}, 
%Wilma Flinstone$^{2}$, and Barney Rubble$^{2}$ \\
%$^{1}$Department of Statistics, University of Warwick, Coventry CV4 7AL, U.K \\
%$^{2}$Department of Geology, University of Bedrock, Bedrock, Kansas, U.S.A.}


\begin{document}

%  This will produce the submission and review information that appears
%  right after the reference section.  Of course, it will be unknown when
%  you submit your paper, so you can either leave this out or put in 
%  sample dates (these will have no effect on the fate of your paper in the
%  review process!)

\date{{\it Received October} 2007. {\it Revised February} 2008.  {\it
Accepted March} 2008.}

%  These options will count the number of pages and provide volume
%  and date information in the upper left hand corner of the top of the 
%  first page as in published papers.  The \pagerange command will only
%  work if you place the command \label{firstpage} near the beginning
%  of the document and \label{lastpage} at the end of the document, as we
%  have done in this template.

%  Again, putting a volume number and date is for your own amusement and
%  has no bearing on what actually happens to your paper!  

\pagerange{\pageref{firstpage}--\pageref{lastpage}} 
\volume{64}
\pubyear{2008}
\artmonth{December}

%  The \doi command is where the DOI for your paper would be placed should it
%  be published.  Again, if you make one up and stick it here, it means 
%  nothing!

\doi{10.1111/j.1541-0420.2005.00454.x}

%  This label and the label ``lastpage'' are used by the \pagerange
%  command above to give the page range for the article.  You may have 
%  to process the document twice to get this to match up with what you 
%  expect.  When using the referee option, this will not count the pages
%  with tables and figures.  

\label{firstpage}

%  put the summary for your paper here

\begin{abstract}
In studies of infant motor development, a crucial research goal is to identify latent classes of infants that experience delayed development, as this is a known risk factor for adverse outcomes later in life. However, there are a number of statistical challenges in modeling infant development: the data are typically skewed, exhibit intermittent missingness, and are highly correlated across the repeated measurements collected during infancy. Using data from the Nurture study, a cohort of over 600 mother-infant pairs followed from pregnancy to 12 months postpartum, we develop a flexible Bayesian latent class model for the analysis infant motor development. Our model has a number of attractive features. First, we adopt the multivariate skew normal distribution with class-specific parameters that accommodate the inherent correlation and skewness in the data. Second, we model the class membership probabilities using a novel Pólya-Gamma data-augmentation scheme, thereby improving predictions of the class membership allocations. Lastly, we impute missing responses under missing at random assumption by drawing from appropriate conditional skew normal distributions. Bayesian inference is achieved through straightforward Gibbs sampling, and can be carried out in available software such as R.  Through simulation studies, we show that the proposed model yields improved inferences over models that ignore skewness. In addition, our imputation method yields improvements compared to conventional missing data methods, including multiple imputation and complete or available case analysis. When applied to Nurture data, we identified two distinct development classes: one characterized by delayed “U-shaped” development and a higher percentage of male infants and another characterized by more steady development and a lower percentage of males. The classes also differed in terms of key demographic variables, such as infant race and maternal pre-pregnancy body mass index. These findings can aid investigators in targeting interventions during this critical early-life developmental window.
\end{abstract}

%  Please place your key words in alphabetical order, separated
%  by semicolons, with the first letter of the first word capitalized,
%  and a period at the end of the list.
%

\begin{keywords}
A key word; But another key word; Still another key word; Yet another key word.
\end{keywords}

%  As usual, the \maketitle command creates the title and author/affiliations
%  display

\maketitle

\setcounter{tocdepth}{3}
\tableofcontents

\newpage

%  If you are using the referee option, a new page, numbered page 1, will
%  start after the summary and keywords.  The page numbers thus count the
%  number of pages of your manuscript in the preferred submission style.
%  Remember, ``Normally, regular papers exceeding 25 pages and Reader Reaction 
%  papers exceeding 12 pages in (the preferred style) will be returned to 
%  the authors without review. The page limit includes acknowledgements, 
%  references, and appendices, but not tables and figures. The page count does 
%  not include the title page and abstract. A maximum of six (6) tables or 
%  figures combined is often required.''

%  You may now place the substance of your manuscript here.  Please use
%  the \section, \subsection, etc commands as described in the user guide.
%  Please use \label and \ref commands to cross-reference sections, equations,
%  tables, figures, etc.
%
%  Please DO NOT attempt to reformat the style of equation numbering!
%  For that matter, please do not attempt to redefine anything!

\section{Introduction}
\label{s:intro}

\section{Nurture Study}
\label{s:nurt}
\subsection{Baseline Demographics and Description of Variables}

\subsection{Statistical Challenges}
\subsubsection{Skewness of Bayley score residuals}
\subsubsection{Attrition and Intermittent Missingness}


\section{Model}
\label{s:model}

\subsection{Multivariate Skew Normal Regression}

We model the effect of covariates on longitudinal development outcomes through the use of a MSN regression model. The MSN distribution can be represented as the superposition of a MN random variable with a latent truncated normal random effect. Let $\mathbf{Y}_{n \times k}$ be the observation matrix such that $Y_{ij}$ is the observation for subject $i$ at timepoint $j$.

$$\mathbf{Y}_{n \times k} = \mathbf{X}_{n \times p}\boldsymbol\beta_{p \times k} + t_{n \times 1}\psi_{1 \times k} + \boldsymbol\epsilon_{n \times k}$$

where $X_i$ is the $1 \times p$ vector of covariate values for subject $i$, $\beta_j$ is the $i \times k$ vector of fixed effects coefficients for timepoint $j$, $t_i \stackrel{iid}{\sim} N_{[0,\infty)}(0,1)$ is a truncated normal random effect, $\psi$ is the vector containing skewness parameters for each timepoint, and $\boldsymbol\epsilon_i \sim N_k(0,\boldsymbol\Sigma_{k \times k})$ is the correlated error term.

\subsection{Multinomial Regression on Class Probabilities}

A primary concern of our model is with identification of latent infant development clusters. We accomplish this via multinomial logit regression model on cluster membership, which utilizes P\'{o}lya-Gamma data-aumentation to allow for updating of all parameters using Gibbs sampling. The multinomial logit model is as follows for $l = 1,...,h$.

$$P(Z_i = l|w_i) = \pi_{il} = \frac{e^{w_i^T \delta_l}}{\sum_{r = 1}^h e^{w_i^T \delta_r}}$$

where $w_i$ is the vector of class probability covariates for subject $i$, $\delta_l$ contains the multinomial regression parameters for class $l$, and $h$ is the number of putative clusters specified \textit{a priori}.

During our MCMC estimation procedure, the class labels $z_i$ are updated from their multinomial full conditional distribution and used in the remaining MCMC steps as class assignments.

\subsection{Conditional MVSN Imputation}

We allow for missingness of outcomes in the MSN mixture model by imputing missing values from their conditional multivariate normal distributions. We note that

$$Y_i|X_i,t_i,\boldsymbol\beta,\psi \sim N_k(X_i \boldsymbol\beta + t_i \psi, \boldsymbol\Sigma)$$

This allows us to appeal to standard conditional forms of the multivariate normal distribution. Let $Y_i = [Y^{miss}_{i_{q \times 1}} | Y^{obs}_{i_{k - q \times 1}}]^T$. We have

$$Y_i^{miss}|Y_i^{obs},X_i,t_i,\boldsymbol\beta,\psi \sim N(\mu^{miss},\boldsymbol\Sigma^{miss})$$

where $\mu^{miss}$ and $\Sigma^{miss}$ take standard forms. Each missing outcome is imputed "online", i.e. once per MCMC iteration. This provides more opportunities to explore the parameter space than multiple imputation and avoids multiplicative run-time scaling in $m$, the number of imputations.

\subsection{Bayesian Inference}

\subsubsection{MCMC Algorithm}

\subsubsection{Assessment of MCMC Convergence}

\subsubsection{Label Switching}

\subsubsection{Prior Choice}

\section{Simulation Studies}
\label{s:sim}

\subsection{Simulation to Compare to Multivariate Normal}

\subsection{Simulation to Compare Imputation Methods}

\subsection{Simulation to Assess Sensitivity to Mispecified H}

\section{Application}
\label{s:app}

\section{Discussion}
\label{s:discuss}

\section{Appendix}

Put your final comments here. 

%  The \backmatter command formats the subsequent headings so that they
%  are in the journal style.  Please keep this command in your document
%  in this position, right after the final section of the main part of 
%  the paper and right before the Acknowledgements, Supplementary Materials,
%  and References sections. 

\backmatter

%  This section is optional.  Here is where you will want to cite
%  grants, people who helped with the paper, etc.  But keep it short!

\section*{Acknowledgements}

%  If your paper refers to supplementary web material, then you MUST
%  include this section!!  See Instructions for Authors at the journal
%  website http://www.biometrics.tibs.org

\section*{Supplementary Materials}

\subsection{Glossary of Notation}

\begin{itemize}

    \item $\mathbf{Y}$: A $n \times p$ matrix containing all multivariate skew-normal regression outcomes such that $y_{ij}$ is the $j^{th}$ outcome observed for subject $i$, where $i = 1,...,n$ and $j = 1,...p$.
    
    \item $\mathbf{X}$: A $n \times m$ matrix containing all multivariate skew-normal regression covariates such that $x_{ij}$ is the $j^{th}$ covariate value for subject $i$.
    
    \item $\mathbf{B}$: A $m \times p$ matrix containing all multivariate skew-normal regression coefficients such that $\mathbf{B} = \left [ \boldsymbol\beta_1,...,\boldsymbol\beta_p \right ]$, where $\beta_{ij}$ is interpreted as the effect of covariate $i$ on outcome $j$ for $i = 1,...,m$ and $j = 1,...,p$.
    
    \item $\mathbf{E}$: A $n \times p$ matrix of error terms in the multivariate skew-normal regression model component. $\mathbf{E}$ is made up of row vectors $\boldsymbol\epsilon_i = (\epsilon_{i1},...,\epsilon_{ip})$, where $ \boldsymbol\epsilon_i \stackrel{iid}{\sim} N_p(0, \boldsymbol\Sigma)$ for $i = 1,...,n$.
    
    \item $\boldsymbol\Sigma$: A $p \times p$ covariance matrix that defines the correlation between the $p$ multivariate normal outcomes. 
    
    \item $\boldsymbol\Omega$: A $p \times p$ covariance scale matrix that defines the correlation between the $p$ multivariate skew-normal outcomes. 
    
    \item $\boldsymbol\psi$: A $p \times 1$ vector containing the skewness parameter for each outcome.
    
    \item $\boldsymbol\alpha$: A $p \times 1$ vector containing the skewness parameter for each outcome.
    
    \item $\mathbf{t}$: A $n \times 1$ vector of truncated normal random effects used in the stochastic representation of the multivariate skew-normal distribution. For $i = 1,...,n$, $t_i \stackrel{iid}{\sim}T_[0,\infty)(0,1)$
    
    \item $\mathbf{X}^*$: A $n \times (m + 1)$ matrix constructed by column binding $\mathbf{t}$ to $\mathbf{X}$
    
    \item $\mathbf{B}^*$: A $(m+1) \times p$ matrix constructed by row binding $\boldsymbol\psi^T$. to $\mathbf{B}$.

\end{itemize}

\subsection{Derivation of Full Conditional Distributions}

\subsubsection{Multivariate Skew-Normal Regression}

Without loss of generality, we derive the full conditional distributions for the multivariate skew-normal regression model component under the assumption that all observations belong to a single cluster. To make the extension to the case where more than one cluster is specified, simply apply these distributional forms to cluster specific parameters and data. 



\subsubsection{Multinomial Logit Regression}

\subsubsection{Multivariate Normal Conditional Imputation}

%  Here, we create the bibliographic entries manually, following the
%  journal style.  If you use this method or use natbib, PLEASE PAY
%  CAREFUL ATTENTION TO THE BIBLIOGRAPHIC STYLE IN A RECENT ISSUE OF
%  THE JOURNAL AND FOLLOW IT!  Failure to follow stylistic conventions
%  just lengthens the time spend copyediting your paper and hence its
%  position in the publication queue should it be accepted.

%  We greatly prefer that you incorporate the references for your
%  article into the body of the article as we have done here 
%  (you can use natbib or not as you choose) than use BiBTeX,
%  so that your article is self-contained in one file.
%  If you do use BiBTeX, please use the .bst file that comes with 
%  the distribution.  In this case, replace the thebibliography
%  environment below by 
%
%  \bibliographystyle{biom} 
% \bibliography{mybibilo.bib}

\begin{thebibliography}{}

\bibitem{ } Cox, D. R. (1972). Regression models and life tables (with
discussion).  \textit{Journal of the Royal Statistical Society, Series B}
\textbf{34,} 187--200.

\bibitem{ }  Hastie, T., Tibshirani, R., and Friedman, J. (2001). \textit{The 
Elements of Statistical Learning: Data Mining, Inference, and Prediction}.
New York: Springer.

\end{thebibliography}

% \appendix

%  To get the journal style of heading for an appendix, mimic the following.


\label{lastpage}

\end{document}
