\documentclass[]{article}
\usepackage{lmodern}
\usepackage{amssymb,amsmath}
\usepackage{ifxetex,ifluatex}
\usepackage{fixltx2e} % provides \textsubscript
\ifnum 0\ifxetex 1\fi\ifluatex 1\fi=0 % if pdftex
  \usepackage[T1]{fontenc}
  \usepackage[utf8]{inputenc}
\else % if luatex or xelatex
  \ifxetex
    \usepackage{mathspec}
  \else
    \usepackage{fontspec}
  \fi
  \defaultfontfeatures{Ligatures=TeX,Scale=MatchLowercase}
\fi
% use upquote if available, for straight quotes in verbatim environments
\IfFileExists{upquote.sty}{\usepackage{upquote}}{}
% use microtype if available
\IfFileExists{microtype.sty}{%
\usepackage{microtype}
\UseMicrotypeSet[protrusion]{basicmath} % disable protrusion for tt fonts
}{}
\usepackage[margin=1in]{geometry}
\usepackage{hyperref}
\hypersetup{unicode=true,
            pdftitle={Simulation Tables},
            pdfauthor={Carter Allen},
            pdfborder={0 0 0},
            breaklinks=true}
\urlstyle{same}  % don't use monospace font for urls
\usepackage{graphicx,grffile}
\makeatletter
\def\maxwidth{\ifdim\Gin@nat@width>\linewidth\linewidth\else\Gin@nat@width\fi}
\def\maxheight{\ifdim\Gin@nat@height>\textheight\textheight\else\Gin@nat@height\fi}
\makeatother
% Scale images if necessary, so that they will not overflow the page
% margins by default, and it is still possible to overwrite the defaults
% using explicit options in \includegraphics[width, height, ...]{}
\setkeys{Gin}{width=\maxwidth,height=\maxheight,keepaspectratio}
\IfFileExists{parskip.sty}{%
\usepackage{parskip}
}{% else
\setlength{\parindent}{0pt}
\setlength{\parskip}{6pt plus 2pt minus 1pt}
}
\setlength{\emergencystretch}{3em}  % prevent overfull lines
\providecommand{\tightlist}{%
  \setlength{\itemsep}{0pt}\setlength{\parskip}{0pt}}
\setcounter{secnumdepth}{0}
% Redefines (sub)paragraphs to behave more like sections
\ifx\paragraph\undefined\else
\let\oldparagraph\paragraph
\renewcommand{\paragraph}[1]{\oldparagraph{#1}\mbox{}}
\fi
\ifx\subparagraph\undefined\else
\let\oldsubparagraph\subparagraph
\renewcommand{\subparagraph}[1]{\oldsubparagraph{#1}\mbox{}}
\fi

%%% Use protect on footnotes to avoid problems with footnotes in titles
\let\rmarkdownfootnote\footnote%
\def\footnote{\protect\rmarkdownfootnote}

%%% Change title format to be more compact
\usepackage{titling}

% Create subtitle command for use in maketitle
\providecommand{\subtitle}[1]{
  \posttitle{
    \begin{center}\large#1\end{center}
    }
}

\setlength{\droptitle}{-2em}

  \title{Simulation Tables}
    \pretitle{\vspace{\droptitle}\centering\huge}
  \posttitle{\par}
    \author{Carter Allen}
    \preauthor{\centering\large\emph}
  \postauthor{\par}
    \date{}
    \predate{}\postdate{}
  
\usepackage{booktabs}
\usepackage{longtable}
\usepackage{array}
\usepackage{multirow}
\usepackage{wrapfig}
\usepackage{float}
\usepackage{colortbl}
\usepackage{pdflscape}
\usepackage{tabu}
\usepackage{threeparttable}
\usepackage{threeparttablex}
\usepackage[normalem]{ulem}
\usepackage{makecell}
\usepackage{xcolor}

\begin{document}
\maketitle

\begin{table}[t]

\caption{\label{tab:unnamed-chunk-5}Model results for simulated data with n = 1,000, J = 4, p = 2, K = 3, r = 2. 1,000 iterations were run with a burn in of 100. Missingness mechanism was MAR and P(miss) = 0. Model results for the multivariate skew normal (MSN) and multivariate normal (MN) mixtures are presented.}
\centering
\fontsize{7}{9}\selectfont
\begin{tabular}{lllll}
\toprule
\multicolumn{2}{c}{ } & \multicolumn{3}{c}{Class 1} \\
\cmidrule(l{3pt}r{3pt}){3-5}
Component & Param. & True & MSN Est. (95\% CrI) & MN Est. (95\% CrI) \\
\midrule
\addlinespace[0.3em]
\multicolumn{5}{l}{\textbf{ }}\\
\hspace{1em}MVSN & $\beta_{11}$ & 11 & 11.07 (10.74, 11.39) & 9.42 (8.91, 9.77)\\
\hspace{1em}Regression & $\beta_{21}$ & 12 & 12.02 (11.87, 12.17) & 11.98 (11.77, 12.18)\\
\hspace{1em} & $\beta_{31}$ & 13 & 13.06 (12.75, 13.36) & 11.39 (10.7, 11.78)\\
\hspace{1em} & $\beta_{41}$ & 14 & 14.06 (13.91, 14.22) & 14.02 (13.78, 14.22)\\
\hspace{1em} & $\beta_{12}$ & 2 & 2.11 (1.82, 2.35) & 0.42 (0.03, 0.83)\\
\hspace{1em} & $\beta_{22}$ & 2 & 2.03 (1.88, 2.17) & 2.02 (1.86, 2.22)\\
\hspace{1em} & $\beta_{32}$ & 2 & 2.13 (1.8, 2.43) & 0.49 (0.14, 0.86)\\
\hspace{1em} & $\beta_{42}$ & 2 & 2.08 (1.93, 2.23) & 2.08 (1.92, 2.28)\\
\addlinespace[0.3em]
\multicolumn{5}{l}{\textbf{ }}\\
\hspace{1em} & $\alpha_{1}$ & -0.99 & -0.81 (-2.12, 0.05) & 0 (0, 0)\\
\hspace{1em} & $\alpha_{2}$ & -0.5 & -0.22 (-1.3, 0.75) & 0 (0, 0)\\
\hspace{1em} & $\alpha_{3}$ & -0.5 & -0.96 (-2.14, 0.01) & 0 (0, 0)\\
\hspace{1em} & $\alpha_{4}$ & -0.99 & -1.18 (-2.44, -0.06) & 0 (0, 0)\\
\addlinespace[0.3em]
\multicolumn{5}{l}{\textbf{ }}\\
\hspace{1em}Multinom. & $\delta_{11}$ & -0.08 & -0.07 (-0.27, 0.12) & -0.54 (-0.77, -0.32)\\
\hspace{1em} & $\delta_{12}$ & 0.51 & 0.25 (-0.04, 0.53) & -0.26 (-0.6, 0.05)\\
\hspace{1em} & $\delta_{21}$ & -0.97 & -0.71 (-0.95, -0.48) & -0.07 (-0.28, 0.14)\\
\hspace{1em} & $\delta_{22}$ & 0.84 & 0.39 (0.09, 0.71) & 0.24 (-0.04, 0.5)\\
\addlinespace[0.3em]
\multicolumn{5}{l}{\textbf{ }}\\
\hspace{1em}Clustering & $\pi_l$ & 0.38 & 0.38 (0.38, 0.38) & 0.38 (0.13, 0.41)\\
\bottomrule
\end{tabular}
\end{table}


\end{document}
