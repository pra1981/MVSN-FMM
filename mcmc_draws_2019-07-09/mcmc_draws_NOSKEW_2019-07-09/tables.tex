\documentclass[]{article}
\usepackage{lmodern}
\usepackage{amssymb,amsmath}
\usepackage{ifxetex,ifluatex}
\usepackage{fixltx2e} % provides \textsubscript
\ifnum 0\ifxetex 1\fi\ifluatex 1\fi=0 % if pdftex
  \usepackage[T1]{fontenc}
  \usepackage[utf8]{inputenc}
\else % if luatex or xelatex
  \ifxetex
    \usepackage{mathspec}
  \else
    \usepackage{fontspec}
  \fi
  \defaultfontfeatures{Ligatures=TeX,Scale=MatchLowercase}
\fi
% use upquote if available, for straight quotes in verbatim environments
\IfFileExists{upquote.sty}{\usepackage{upquote}}{}
% use microtype if available
\IfFileExists{microtype.sty}{%
\usepackage{microtype}
\UseMicrotypeSet[protrusion]{basicmath} % disable protrusion for tt fonts
}{}
\usepackage[margin=1in]{geometry}
\usepackage{hyperref}
\hypersetup{unicode=true,
            pdftitle={Simulation Tables},
            pdfauthor={Carter Allen},
            pdfborder={0 0 0},
            breaklinks=true}
\urlstyle{same}  % don't use monospace font for urls
\usepackage{graphicx,grffile}
\makeatletter
\def\maxwidth{\ifdim\Gin@nat@width>\linewidth\linewidth\else\Gin@nat@width\fi}
\def\maxheight{\ifdim\Gin@nat@height>\textheight\textheight\else\Gin@nat@height\fi}
\makeatother
% Scale images if necessary, so that they will not overflow the page
% margins by default, and it is still possible to overwrite the defaults
% using explicit options in \includegraphics[width, height, ...]{}
\setkeys{Gin}{width=\maxwidth,height=\maxheight,keepaspectratio}
\IfFileExists{parskip.sty}{%
\usepackage{parskip}
}{% else
\setlength{\parindent}{0pt}
\setlength{\parskip}{6pt plus 2pt minus 1pt}
}
\setlength{\emergencystretch}{3em}  % prevent overfull lines
\providecommand{\tightlist}{%
  \setlength{\itemsep}{0pt}\setlength{\parskip}{0pt}}
\setcounter{secnumdepth}{0}
% Redefines (sub)paragraphs to behave more like sections
\ifx\paragraph\undefined\else
\let\oldparagraph\paragraph
\renewcommand{\paragraph}[1]{\oldparagraph{#1}\mbox{}}
\fi
\ifx\subparagraph\undefined\else
\let\oldsubparagraph\subparagraph
\renewcommand{\subparagraph}[1]{\oldsubparagraph{#1}\mbox{}}
\fi

%%% Use protect on footnotes to avoid problems with footnotes in titles
\let\rmarkdownfootnote\footnote%
\def\footnote{\protect\rmarkdownfootnote}

%%% Change title format to be more compact
\usepackage{titling}

% Create subtitle command for use in maketitle
\providecommand{\subtitle}[1]{
  \posttitle{
    \begin{center}\large#1\end{center}
    }
}

\setlength{\droptitle}{-2em}

  \title{Simulation Tables}
    \pretitle{\vspace{\droptitle}\centering\huge}
  \posttitle{\par}
    \author{Carter Allen}
    \preauthor{\centering\large\emph}
  \postauthor{\par}
    \date{}
    \predate{}\postdate{}
  
\usepackage{booktabs}
\usepackage{longtable}
\usepackage{array}
\usepackage{multirow}
\usepackage{wrapfig}
\usepackage{float}
\usepackage{colortbl}
\usepackage{pdflscape}
\usepackage{tabu}
\usepackage{threeparttable}
\usepackage{threeparttablex}
\usepackage[normalem]{ulem}
\usepackage{makecell}
\usepackage{xcolor}

\begin{document}
\maketitle

\begin{table}[t]

\caption{\label{tab:unnamed-chunk-5}Model results for simulated data with n = 1000, k = 4, p = 2, h = 3, r = 2. 1000 iterations were run with a burn in of 250. Missingness mechanism was MAR and P(miss) = 0}
\centering
\fontsize{8}{10}\selectfont
\begin{tabular}{llllllll}
\toprule
\multicolumn{2}{c}{ } & \multicolumn{2}{c}{Class 1} & \multicolumn{2}{c}{Class 2} & \multicolumn{2}{c}{Class 3} \\
\cmidrule(l{3pt}r{3pt}){3-4} \cmidrule(l{3pt}r{3pt}){5-6} \cmidrule(l{3pt}r{3pt}){7-8}
Model Component & Parameter & True & Est. (95\% CrI) & True & Est. (95\% CrI) & True & Est. (95\% CrI)\\
\midrule
\addlinespace[0.3em]
\multicolumn{8}{l}{\textbf{ }}\\
\hspace{1em}MVSN & $\beta_{11}$ & 0.87 & -5 (-5.22, -4.76) & 3.04 & 10.92 (10.68, 11.16) & 6.96 & -9.83 (-10.2, -9.42)\\
\hspace{1em}Regression & $\beta_{21}$ & 2.24 & -3.99 (-4.09, -3.88) & 3.08 & 12.02 (11.88, 12.16) & 8.35 & -10.84 (-11.05, -10.63)\\
\hspace{1em} & $\beta_{31}$ & 2.47 & -2.91 (-3.12, -2.69) & 4.08 & 13 (12.73, 13.29) & 9.73 & -11.89 (-12.32, -11.45)\\
\hspace{1em} & $\beta_{41}$ & -0.18 & -2 (-2.1, -1.91) & 3.71 & 13.98 (13.81, 14.12) & 9.16 & -12.8 (-13.05, -12.55)\\
\hspace{1em} & $\beta_{12}$ & 0.11 & 4.94 (4.71, 5.17) & 3.97 & 1.98 (1.77, 2.18) & 8.85 & -1.94 (-2.25, -1.62)\\
\hspace{1em} & $\beta_{22}$ & 2.09 & 5 (4.88, 5.1) & 3.8 & 2 (1.9, 2.1) & 9.12 & -1.83 (-2, -1.66)\\
\hspace{1em} & $\beta_{32}$ & 0.13 & 4.88 (4.67, 5.11) & 2.82 & 1.94 (1.74, 2.16) & 8.58 & -1.98 (-2.28, -1.69)\\
\hspace{1em} & $\beta_{42}$ & 1.23 & 4.97 (4.86, 5.08) & 5.05 & 1.92 (1.82, 2.02) & 6.26 & -2 (-2.14, -1.86)\\
\addlinespace[0.3em]
\multicolumn{8}{l}{\textbf{ }}\\
\hspace{1em} & $\Omega_{11}$ & 23.5 & 1.04 (0.89, 1.24) & 77.17 & 1.63 (1.42, 1.89) & 591.27 & 2.14 (1.77, 2.67)\\
\hspace{1em} & $\Omega_{12}$ & 10.83 & 0.49 (0.37, 0.62) & 79.89 & 1.24 (1.04, 1.49) & 580.88 & 2 (1.59, 2.6)\\
\hspace{1em} & $\Omega_{13}$ & 13.41 & 0.08 (-0.03, 0.2) & 69.23 & 0.33 (0.2, 0.47) & 584.5 & 0.53 (0.29, 0.82)\\
\hspace{1em} & $\Omega_{14}$ & 6.25 & -0.02 (-0.15, 0.09) & 62.8 & 0.17 (0.04, 0.31) & 575.71 & 0.38 (0.18, 0.64)\\
\hspace{1em} & $\Omega_{22}$ & 9.97 & 0.99 (0.86, 1.16) & 88.8 & 1.95 (1.71, 2.23) & 578.02 & 2.9 (2.38, 3.63)\\
\hspace{1em} & $\Omega_{23}$ & 5.44 & 0.47 (0.34, 0.61) & 75.11 & 0.64 (0.5, 0.8) & 579.01 & 0.92 (0.61, 1.28)\\
\hspace{1em} & $\Omega_{24}$ & 5.72 & 0.13 (0.01, 0.25) & 63.9 & 0.38 (0.24, 0.54) & 575.04 & 0.64 (0.4, 0.95)\\
\hspace{1em} & $\Omega_{33}$ & 14.61 & 1.19 (1.03, 1.4) & 71.41 & 1.01 (0.88, 1.16) & 589.23 & 1.2 (0.97, 1.49)\\
\hspace{1em} & $\Omega_{34}$ & 7.75 & 0.65 (0.52, 0.81) & 65.11 & 0.51 (0.41, 0.64) & 576.78 & 0.63 (0.47, 0.82)\\
\hspace{1em} & $\Omega_{44}$ & 10.53 & 1.12 (0.97, 1.34) & 64.59 & 0.99 (0.86, 1.14) & 578.51 & 1.01 (0.82, 1.23)\\
\addlinespace[0.3em]
\multicolumn{8}{l}{\textbf{ }}\\
\hspace{1em} & $\alpha_{1}$ & 0.68 & 0 (0, 0) & 24.14 & 0 (0, 0) & 18.65 & 0 (0, 0)\\
\hspace{1em} & $\alpha_{2}$ & -1.01 & 0 (0, 0) & 9.6 & 0 (0, 0) & -70.21 & 0 (0, 0)\\
\hspace{1em} & $\alpha_{3}$ & -0.7 & 0 (0, 0) & -51.02 & 0 (0, 0) & 5.37 & 0 (0, 0)\\
\hspace{1em} & $\alpha_{4}$ & -0.2 & 0 (0, 0) & 138.62 & 0 (0, 0) & -29.34 & 0 (0, 0)\\
\addlinespace[0.3em]
\multicolumn{8}{l}{\textbf{ }}\\
\hspace{1em}Multinom. & $\delta_{11}$ & 0 & 0.09 (-0.12, 0.32) & 0 & 0.09 (-0.12, 0.32) & 0 & 0.09 (-0.12, 0.32)\\
\hspace{1em} & $\delta_{12}$ & 0.86 & 0.79 (0.49, 1.08) & 0.86 & 0.79 (0.49, 1.08) & 0.86 & 0.79 (0.49, 1.08)\\
\hspace{1em} & $\delta_{21}$ & -0.01 & 0.03 (-0.17, 0.26) & -0.01 & 0.03 (-0.17, 0.26) & -0.01 & 0.03 (-0.17, 0.26)\\
\hspace{1em} & $\delta_{22}$ & -0.66 & -0.59 (-0.95, -0.21) & -0.66 & -0.59 (-0.95, -0.21) & -0.66 & -0.59 (-0.95, -0.21)\\
\addlinespace[0.3em]
\multicolumn{8}{l}{\textbf{ }}\\
\hspace{1em}Clustering & $\pi_l$ & 0.28 & 0.37 (0.34, 0.4) & 0.48 & 0.42 (0.39, 0.45) & 0.23 & 0.21 (0.18, 0.23)\\
\bottomrule
\end{tabular}
\end{table}


\end{document}
