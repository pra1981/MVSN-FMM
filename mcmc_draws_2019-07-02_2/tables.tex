\documentclass[]{article}
\usepackage{lmodern}
\usepackage{amssymb,amsmath}
\usepackage{ifxetex,ifluatex}
\usepackage{fixltx2e} % provides \textsubscript
\ifnum 0\ifxetex 1\fi\ifluatex 1\fi=0 % if pdftex
  \usepackage[T1]{fontenc}
  \usepackage[utf8]{inputenc}
\else % if luatex or xelatex
  \ifxetex
    \usepackage{mathspec}
  \else
    \usepackage{fontspec}
  \fi
  \defaultfontfeatures{Ligatures=TeX,Scale=MatchLowercase}
\fi
% use upquote if available, for straight quotes in verbatim environments
\IfFileExists{upquote.sty}{\usepackage{upquote}}{}
% use microtype if available
\IfFileExists{microtype.sty}{%
\usepackage{microtype}
\UseMicrotypeSet[protrusion]{basicmath} % disable protrusion for tt fonts
}{}
\usepackage[margin=1in]{geometry}
\usepackage{hyperref}
\hypersetup{unicode=true,
            pdftitle={Simulation Tables},
            pdfauthor={Carter Allen},
            pdfborder={0 0 0},
            breaklinks=true}
\urlstyle{same}  % don't use monospace font for urls
\usepackage{graphicx,grffile}
\makeatletter
\def\maxwidth{\ifdim\Gin@nat@width>\linewidth\linewidth\else\Gin@nat@width\fi}
\def\maxheight{\ifdim\Gin@nat@height>\textheight\textheight\else\Gin@nat@height\fi}
\makeatother
% Scale images if necessary, so that they will not overflow the page
% margins by default, and it is still possible to overwrite the defaults
% using explicit options in \includegraphics[width, height, ...]{}
\setkeys{Gin}{width=\maxwidth,height=\maxheight,keepaspectratio}
\IfFileExists{parskip.sty}{%
\usepackage{parskip}
}{% else
\setlength{\parindent}{0pt}
\setlength{\parskip}{6pt plus 2pt minus 1pt}
}
\setlength{\emergencystretch}{3em}  % prevent overfull lines
\providecommand{\tightlist}{%
  \setlength{\itemsep}{0pt}\setlength{\parskip}{0pt}}
\setcounter{secnumdepth}{0}
% Redefines (sub)paragraphs to behave more like sections
\ifx\paragraph\undefined\else
\let\oldparagraph\paragraph
\renewcommand{\paragraph}[1]{\oldparagraph{#1}\mbox{}}
\fi
\ifx\subparagraph\undefined\else
\let\oldsubparagraph\subparagraph
\renewcommand{\subparagraph}[1]{\oldsubparagraph{#1}\mbox{}}
\fi

%%% Use protect on footnotes to avoid problems with footnotes in titles
\let\rmarkdownfootnote\footnote%
\def\footnote{\protect\rmarkdownfootnote}

%%% Change title format to be more compact
\usepackage{titling}

% Create subtitle command for use in maketitle
\providecommand{\subtitle}[1]{
  \posttitle{
    \begin{center}\large#1\end{center}
    }
}

\setlength{\droptitle}{-2em}

  \title{Simulation Tables}
    \pretitle{\vspace{\droptitle}\centering\huge}
  \posttitle{\par}
    \author{Carter Allen}
    \preauthor{\centering\large\emph}
  \postauthor{\par}
    \date{}
    \predate{}\postdate{}
  
\usepackage{booktabs}
\usepackage{longtable}
\usepackage{array}
\usepackage{multirow}
\usepackage{wrapfig}
\usepackage{float}
\usepackage{colortbl}
\usepackage{pdflscape}
\usepackage{tabu}
\usepackage{threeparttable}
\usepackage{threeparttablex}
\usepackage[normalem]{ulem}
\usepackage{makecell}
\usepackage{xcolor}

\begin{document}
\maketitle

\begin{table}[t]

\caption{\label{tab:unnamed-chunk-4}Model results for simulated data with n = 1000, k = 4, p = 2, h = 3, r = 2. 1000 iterations were run with a burn in of 250. Missingness mechanism was MAR and P(miss) = 0}
\centering
\fontsize{8}{10}\selectfont
\begin{tabular}{llllllll}
\toprule
\multicolumn{2}{c}{ } & \multicolumn{2}{c}{Class 1} & \multicolumn{2}{c}{Class 2} & \multicolumn{2}{c}{Class 3} \\
\cmidrule(l{3pt}r{3pt}){3-4} \cmidrule(l{3pt}r{3pt}){5-6} \cmidrule(l{3pt}r{3pt}){7-8}
Model Component & Parameter & True & Est. (95\% CrI) & True & Est. (95\% CrI) & True & Est. (95\% CrI)\\
\midrule
\addlinespace[0.3em]
\multicolumn{8}{l}{\textbf{ }}\\
\hspace{1em}MVSN & $\beta_{11}$ & -2.13 & -2.78 (-3.35, -2.12) & 0.36 & 0.49 (-0.08, 1.25) & 2.98 & 1.95 (1.21, 2.87)\\
\hspace{1em}Regression & $\beta_{21}$ & -2.94 & -2.91 (-3.07, -2.75) & -0.06 & -0.11 (-0.23, 0.01) & 3.61 & 3.5 (3.3, 3.66)\\
\hspace{1em} & $\beta_{31}$ & -2.05 & -2.88 (-3.4, -2.03) & -0.31 & -0.05 (-0.73, 0.58) & 3.22 & 2.62 (1.5, 3.38)\\
\hspace{1em} & $\beta_{41}$ & -3.06 & -3.04 (-3.18, -2.89) & -0.87 & -0.9 (-1.01, -0.79) & 3.71 & 3.62 (3.39, 3.8)\\
\hspace{1em} & $\beta_{12}$ & -3.08 & -3.44 (-3.92, -2.94) & -0.71 & -0.27 (-0.9, 0.23) & 3.36 & 2.71 (1.55, 3.59)\\
\hspace{1em} & $\beta_{22}$ & -3.47 & -3.36 (-3.51, -3.21) & 0.4 & 0.27 (0.15, 0.4) & 3.35 & 3.19 (2.97, 3.38)\\
\hspace{1em} & $\beta_{32}$ & -2.16 & -2.35 (-3.11, -1.87) & -0.06 & 0.37 (-0.33, 0.94) & 3.06 & 2.47 (1.37, 3.21)\\
\hspace{1em} & $\beta_{42}$ & -3.05 & -2.99 (-3.13, -2.85) & -0.33 & -0.41 (-0.52, -0.29) & 3.01 & 2.89 (2.68, 3.08)\\
\addlinespace[0.3em]
\multicolumn{8}{l}{\textbf{ }}\\
\hspace{1em} & $\sigma_{11}$ & 1 & 1.41 (1.1, 1.75) & 1 & 1.06 (0.76, 1.38) & 1 & 1.44 (0.95, 2.09)\\
\hspace{1em} & $\sigma_{12}$ & 0.5 & 0.78 (0.51, 1.08) & 0.5 & 0.45 (0.2, 0.69) & 0.5 & 1.2 (0.71, 1.88)\\
\hspace{1em} & $\sigma_{13}$ & 0.25 & 0.59 (0.38, 0.86) & 0.25 & 0.28 (0.05, 0.54) & 0.25 & 1.01 (0.53, 1.72)\\
\hspace{1em} & $\sigma_{14}$ & 0.12 & 0.39 (0.18, 0.64) & 0.12 & 0.16 (-0.04, 0.4) & 0.12 & 0.82 (0.37, 1.42)\\
\hspace{1em} & $\sigma_{22}$ & 1 & 1.15 (0.8, 1.48) & 1 & 0.86 (0.53, 1.15) & 1 & 1.68 (1.15, 2.52)\\
\hspace{1em} & $\sigma_{23}$ & 0.5 & 0.75 (0.52, 1) & 0.5 & 0.41 (0.13, 0.67) & 0.5 & 1.33 (0.82, 2.13)\\
\hspace{1em} & $\sigma_{24}$ & 0.25 & 0.5 (0.26, 0.76) & 0.25 & 0.18 (-0.05, 0.41) & 0.25 & 1.04 (0.55, 1.73)\\
\hspace{1em} & $\sigma_{33}$ & 1 & 1.18 (0.91, 1.49) & 1 & 0.93 (0.67, 1.29) & 1 & 1.89 (1.34, 2.83)\\
\hspace{1em} & $\sigma_{34}$ & 0.5 & 0.75 (0.51, 1.01) & 0.5 & 0.44 (0.2, 0.74) & 0.5 & 1.32 (0.8, 2.08)\\
\hspace{1em} & $\sigma_{44}$ & 1 & 1.17 (0.87, 1.47) & 1 & 0.92 (0.61, 1.22) & 1 & 1.68 (1.1, 2.37)\\
\addlinespace[0.3em]
\multicolumn{8}{l}{\textbf{ }}\\
\hspace{1em} & $\psi_{1}$ & -0.67 & 0.12 (-0.7, 0.8) & 0.33 & 0.04 (-0.97, 0.75) & -1.33 & -0.29 (-1.73, 0.66)\\
\hspace{1em} & $\psi_{2}$ & -0.67 & 0.39 (-0.68, 1.01) & 0.33 & -0.04 (-0.89, 0.82) & -1.33 & -0.71 (-1.94, 0.7)\\
\hspace{1em} & $\psi_{3}$ & -0.67 & -0.26 (-0.87, 0.3) & 0.33 & -0.18 (-0.81, 0.61) & -1.33 & -0.6 (-2.02, 0.84)\\
\hspace{1em} & $\psi_{4}$ & -0.67 & -0.44 (-1.01, 0.52) & 0.33 & -0.17 (-0.91, 0.71) & -1.33 & -0.67 (-1.85, 0.69)\\
\addlinespace[0.3em]
\multicolumn{8}{l}{\textbf{ }}\\
\hspace{1em}Multinom. & $\delta_{11}$ & 0.55 & 0.42 (0.23, 0.62) & 0.55 & 0.42 (0.23, 0.62) & 0.55 & 0.42 (0.23, 0.62)\\
\hspace{1em} & $\delta_{12}$ & -0.51 & -0.52 (-0.82, -0.23) & -0.51 & -0.52 (-0.82, -0.23) & -0.51 & -0.52 (-0.82, -0.23)\\
\hspace{1em} & $\delta_{21}$ & -0.29 & -0.1 (-0.32, 0.12) & -0.29 & -0.1 (-0.32, 0.12) & -0.29 & -0.1 (-0.32, 0.12)\\
\hspace{1em} & $\delta_{22}$ & 0.19 & -0.04 (-0.35, 0.27) & 0.19 & -0.04 (-0.35, 0.27) & 0.19 & -0.04 (-0.35, 0.27)\\
\addlinespace[0.3em]
\multicolumn{8}{l}{\textbf{ }}\\
\hspace{1em}Clustering & $\pi_l$ & 0.32 & 0.32 (0.28, 0.35) & 0.39 & 0.4 (0.35, 0.43) & 0.29 & 0.29 (0.26, 0.32)\\
\bottomrule
\end{tabular}
\end{table}


\end{document}
