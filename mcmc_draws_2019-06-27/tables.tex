\documentclass[]{article}
\usepackage{lmodern}
\usepackage{amssymb,amsmath}
\usepackage{ifxetex,ifluatex}
\usepackage{fixltx2e} % provides \textsubscript
\ifnum 0\ifxetex 1\fi\ifluatex 1\fi=0 % if pdftex
  \usepackage[T1]{fontenc}
  \usepackage[utf8]{inputenc}
\else % if luatex or xelatex
  \ifxetex
    \usepackage{mathspec}
  \else
    \usepackage{fontspec}
  \fi
  \defaultfontfeatures{Ligatures=TeX,Scale=MatchLowercase}
\fi
% use upquote if available, for straight quotes in verbatim environments
\IfFileExists{upquote.sty}{\usepackage{upquote}}{}
% use microtype if available
\IfFileExists{microtype.sty}{%
\usepackage{microtype}
\UseMicrotypeSet[protrusion]{basicmath} % disable protrusion for tt fonts
}{}
\usepackage[margin=1in]{geometry}
\usepackage{hyperref}
\hypersetup{unicode=true,
            pdftitle={Simulation Tables},
            pdfauthor={Carter Allen},
            pdfborder={0 0 0},
            breaklinks=true}
\urlstyle{same}  % don't use monospace font for urls
\usepackage{graphicx,grffile}
\makeatletter
\def\maxwidth{\ifdim\Gin@nat@width>\linewidth\linewidth\else\Gin@nat@width\fi}
\def\maxheight{\ifdim\Gin@nat@height>\textheight\textheight\else\Gin@nat@height\fi}
\makeatother
% Scale images if necessary, so that they will not overflow the page
% margins by default, and it is still possible to overwrite the defaults
% using explicit options in \includegraphics[width, height, ...]{}
\setkeys{Gin}{width=\maxwidth,height=\maxheight,keepaspectratio}
\IfFileExists{parskip.sty}{%
\usepackage{parskip}
}{% else
\setlength{\parindent}{0pt}
\setlength{\parskip}{6pt plus 2pt minus 1pt}
}
\setlength{\emergencystretch}{3em}  % prevent overfull lines
\providecommand{\tightlist}{%
  \setlength{\itemsep}{0pt}\setlength{\parskip}{0pt}}
\setcounter{secnumdepth}{0}
% Redefines (sub)paragraphs to behave more like sections
\ifx\paragraph\undefined\else
\let\oldparagraph\paragraph
\renewcommand{\paragraph}[1]{\oldparagraph{#1}\mbox{}}
\fi
\ifx\subparagraph\undefined\else
\let\oldsubparagraph\subparagraph
\renewcommand{\subparagraph}[1]{\oldsubparagraph{#1}\mbox{}}
\fi

%%% Use protect on footnotes to avoid problems with footnotes in titles
\let\rmarkdownfootnote\footnote%
\def\footnote{\protect\rmarkdownfootnote}

%%% Change title format to be more compact
\usepackage{titling}

% Create subtitle command for use in maketitle
\providecommand{\subtitle}[1]{
  \posttitle{
    \begin{center}\large#1\end{center}
    }
}

\setlength{\droptitle}{-2em}

  \title{Simulation Tables}
    \pretitle{\vspace{\droptitle}\centering\huge}
  \posttitle{\par}
    \author{Carter Allen}
    \preauthor{\centering\large\emph}
  \postauthor{\par}
    \date{}
    \predate{}\postdate{}
  
\usepackage{booktabs}
\usepackage{longtable}
\usepackage{array}
\usepackage{multirow}
\usepackage{wrapfig}
\usepackage{float}
\usepackage{colortbl}
\usepackage{pdflscape}
\usepackage{tabu}
\usepackage{threeparttable}
\usepackage{threeparttablex}
\usepackage[normalem]{ulem}
\usepackage{makecell}
\usepackage{xcolor}

\begin{document}
\maketitle

\begin{table}[t]

\caption{\label{tab:unnamed-chunk-4}Model results for simulated data with n = 1000, k = 4, p = 2, h = 3, r = 2. 1000 iterations were run with a burn in of 250. Missingness mechanism was MAR and P(miss) = 0}
\centering
\fontsize{8}{10}\selectfont
\begin{tabular}{llllllll}
\toprule
\multicolumn{2}{c}{ } & \multicolumn{2}{c}{Class 1} & \multicolumn{2}{c}{Class 2} & \multicolumn{2}{c}{Class 3} \\
\cmidrule(l{3pt}r{3pt}){3-4} \cmidrule(l{3pt}r{3pt}){5-6} \cmidrule(l{3pt}r{3pt}){7-8}
Model Component & Parameter & True & Est. (95\% CrI) & True & Est. (95\% CrI) & True & Est. (95\% CrI)\\
\midrule
\addlinespace[0.3em]
\multicolumn{8}{l}{\textbf{ }}\\
\hspace{1em}MVSN & $\beta_{11}$ & -3.32 & -3.06 (-3.97, -2.01) & 0.48 & 0.84 (-0.46, 1.99) & 2.67 & 1.58 (0.81, 2.34)\\
\hspace{1em}Regression & $\beta_{21}$ & -2.77 & -2.85 (-3.41, -2.24) & 0.52 & 0.6 (-0.18, 1.29) & 3.59 & 3.24 (2.88, 3.53)\\
\hspace{1em} & $\beta_{31}$ & -3.15 & -3.11 (-4.19, -2.17) & 0.88 & 1.01 (-0.33, 2.29) & 2.36 & 1.21 (0.34, 1.92)\\
\hspace{1em} & $\beta_{41}$ & -2.36 & -2.24 (-2.79, -1.65) & -0.29 & -0.31 (-1.13, 0.41) & 3.27 & 2.98 (2.65, 3.29)\\
\hspace{1em} & $\beta_{12}$ & -3.02 & -2.94 (-4.23, -1.89) & -0.74 & -0.19 (-1.43, 1.26) & 3.62 & 2.42 (1.56, 3.87)\\
\hspace{1em} & $\beta_{22}$ & -2.98 & -2.97 (-3.66, -2.21) & -0.56 & -0.4 (-1.19, 0.31) & 2.42 & 2.26 (1.9, 2.67)\\
\hspace{1em} & $\beta_{32}$ & -3 & -2.86 (-4.11, -1.82) & 0.13 & 0.6 (-0.79, 1.7) & 3.16 & 1.92 (1.23, 2.54)\\
\hspace{1em} & $\beta_{42}$ & -4.02 & -3.76 (-4.36, -3.25) & 0.01 & -0.1 (-0.8, 0.67) & 3.48 & 3.26 (2.95, 3.64)\\
\addlinespace[0.3em]
\multicolumn{8}{l}{\textbf{ }}\\
\hspace{1em} & $\sigma_{11}$ & 1 & 1.48 (0.85, 2.78) & 1 & 1.46 (0.56, 3.75) & 1 & 1.12 (0.76, 1.71)\\
\hspace{1em} & $\sigma_{12}$ & 0.57 & 1.3 (0.75, 2.84) & 0.14 & 1.14 (0.33, 3.42) & 0.79 & 0.94 (0.58, 1.5)\\
\hspace{1em} & $\sigma_{13}$ & 0.68 & 1.5 (0.87, 3.16) & 0.25 & 0.76 (0.02, 3.03) & -0.03 & 0.42 (0, 1.08)\\
\hspace{1em} & $\sigma_{14}$ & 0.3 & 1.23 (0.64, 2.13) & 0.61 & 1.02 (0.2, 3.18) & 0.36 & 0.7 (0.32, 1.17)\\
\hspace{1em} & $\sigma_{22}$ & 1 & 1.9 (1.08, 3.35) & 1 & 1.8 (0.69, 3.86) & 1 & 1.13 (0.74, 1.64)\\
\hspace{1em} & $\sigma_{23}$ & 0.9 & 2.18 (1.11, 3.6) & 0.19 & 0.36 (-0.88, 2.39) & -0.13 & 0.26 (-0.2, 0.89)\\
\hspace{1em} & $\sigma_{24}$ & 0.42 & 0.87 (0.27, 2.01) & 0.7 & 1.34 (0.31, 3.68) & 0.82 & 1.1 (0.73, 1.65)\\
\hspace{1em} & $\sigma_{33}$ & 1 & 2.48 (1.3, 4.23) & 1 & 1.67 (0.9, 4.49) & 1 & 2.03 (1.36, 3.29)\\
\hspace{1em} & $\sigma_{34}$ & 0.11 & 0.92 (0.25, 2.08) & -0.12 & -0.02 (-1.51, 1.23) & 0.08 & 0.64 (0.16, 1.37)\\
\hspace{1em} & $\sigma_{44}$ & 1 & 1.75 (1.04, 2.91) & 1 & 1.44 (0.55, 3.8) & 1 & 1.41 (1.01, 2.12)\\
\addlinespace[0.3em]
\multicolumn{8}{l}{\textbf{ }}\\
\hspace{1em} & $\psi_{1}$ & -0.33 & -0.51 (-1.53, 0.38) & 0.67 & 0.28 (-0.92, 1.65) & -1 & 0.15 (-0.57, 1.07)\\
\hspace{1em} & $\psi_{2}$ & -0.33 & -0.31 (-1.39, 0.91) & 0.67 & 0.43 (-1.21, 1.7) & -1 & 0.18 (-0.62, 1.07)\\
\hspace{1em} & $\psi_{3}$ & -0.33 & -0.3 (-1.65, 1.13) & 0.67 & -0.06 (-1.7, 1.26) & -1 & 0.66 (-0.93, 1.47)\\
\hspace{1em} & $\psi_{4}$ & -0.33 & -0.5 (-1.6, 0.84) & 0.67 & 0.05 (-0.85, 1.71) & -1 & 0.34 (-0.35, 1.01)\\
\addlinespace[0.3em]
\multicolumn{8}{l}{\textbf{ }}\\
\hspace{1em}Multinom. & $\delta_{11}$ & -0.87 & -0.36 (-1.27, 0.27) & -0.87 & -0.36 (-1.27, 0.27) & -0.87 & -0.36 (-1.27, 0.27)\\
\hspace{1em} & $\delta_{12}$ & 0.22 & -0.33 (-1.5, 0.96) & 0.22 & -0.33 (-1.5, 0.96) & 0.22 & -0.33 (-1.5, 0.96)\\
\hspace{1em} & $\delta_{21}$ & 0.6 & 0.98 (0.34, 1.48) & 0.6 & 0.98 (0.34, 1.48) & 0.6 & 0.98 (0.34, 1.48)\\
\hspace{1em} & $\delta_{22}$ & -0.2 & -0.19 (-0.87, 0.74) & -0.2 & -0.19 (-0.87, 0.74) & -0.2 & -0.19 (-0.87, 0.74)\\
\addlinespace[0.3em]
\multicolumn{8}{l}{\textbf{ }}\\
\hspace{1em}Clustering & $\pi_l$ & 0.26 & 0.27 (0.24, 0.3) & 0.14 & 0.14 (0.12, 0.17) & 0.6 & 0.59 (0.55, 0.62)\\
\bottomrule
\end{tabular}
\end{table}


\end{document}
