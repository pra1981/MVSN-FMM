\documentclass[]{article}
\usepackage{lmodern}
\usepackage{amssymb,amsmath}
\usepackage{ifxetex,ifluatex}
\usepackage{fixltx2e} % provides \textsubscript
\ifnum 0\ifxetex 1\fi\ifluatex 1\fi=0 % if pdftex
  \usepackage[T1]{fontenc}
  \usepackage[utf8]{inputenc}
\else % if luatex or xelatex
  \ifxetex
    \usepackage{mathspec}
  \else
    \usepackage{fontspec}
  \fi
  \defaultfontfeatures{Ligatures=TeX,Scale=MatchLowercase}
\fi
% use upquote if available, for straight quotes in verbatim environments
\IfFileExists{upquote.sty}{\usepackage{upquote}}{}
% use microtype if available
\IfFileExists{microtype.sty}{%
\usepackage{microtype}
\UseMicrotypeSet[protrusion]{basicmath} % disable protrusion for tt fonts
}{}
\usepackage[margin=1in]{geometry}
\usepackage{hyperref}
\hypersetup{unicode=true,
            pdftitle={Simulation Tables},
            pdfauthor={Carter Allen},
            pdfborder={0 0 0},
            breaklinks=true}
\urlstyle{same}  % don't use monospace font for urls
\usepackage{graphicx,grffile}
\makeatletter
\def\maxwidth{\ifdim\Gin@nat@width>\linewidth\linewidth\else\Gin@nat@width\fi}
\def\maxheight{\ifdim\Gin@nat@height>\textheight\textheight\else\Gin@nat@height\fi}
\makeatother
% Scale images if necessary, so that they will not overflow the page
% margins by default, and it is still possible to overwrite the defaults
% using explicit options in \includegraphics[width, height, ...]{}
\setkeys{Gin}{width=\maxwidth,height=\maxheight,keepaspectratio}
\IfFileExists{parskip.sty}{%
\usepackage{parskip}
}{% else
\setlength{\parindent}{0pt}
\setlength{\parskip}{6pt plus 2pt minus 1pt}
}
\setlength{\emergencystretch}{3em}  % prevent overfull lines
\providecommand{\tightlist}{%
  \setlength{\itemsep}{0pt}\setlength{\parskip}{0pt}}
\setcounter{secnumdepth}{0}
% Redefines (sub)paragraphs to behave more like sections
\ifx\paragraph\undefined\else
\let\oldparagraph\paragraph
\renewcommand{\paragraph}[1]{\oldparagraph{#1}\mbox{}}
\fi
\ifx\subparagraph\undefined\else
\let\oldsubparagraph\subparagraph
\renewcommand{\subparagraph}[1]{\oldsubparagraph{#1}\mbox{}}
\fi

%%% Use protect on footnotes to avoid problems with footnotes in titles
\let\rmarkdownfootnote\footnote%
\def\footnote{\protect\rmarkdownfootnote}

%%% Change title format to be more compact
\usepackage{titling}

% Create subtitle command for use in maketitle
\providecommand{\subtitle}[1]{
  \posttitle{
    \begin{center}\large#1\end{center}
    }
}

\setlength{\droptitle}{-2em}

  \title{Simulation Tables}
    \pretitle{\vspace{\droptitle}\centering\huge}
  \posttitle{\par}
    \author{Carter Allen}
    \preauthor{\centering\large\emph}
  \postauthor{\par}
      \predate{\centering\large\emph}
  \postdate{\par}
    \date{4/11/2019}

\usepackage{booktabs}
\usepackage{longtable}
\usepackage{array}
\usepackage{multirow}
\usepackage{wrapfig}
\usepackage{float}
\usepackage{colortbl}
\usepackage{pdflscape}
\usepackage{tabu}
\usepackage{threeparttable}
\usepackage{threeparttablex}
\usepackage[normalem]{ulem}
\usepackage{makecell}
\usepackage{xcolor}

\begin{document}
\maketitle

\begin{table}[t]

\caption{\label{tab:unnamed-chunk-4}Model results for simulated data with n = 1500, k = 2, p = 2, h = 3, v = 1. 5000 iterations were run with a burn in of 1000. Missingness mechanism was MAR and P(miss) = 0}
\centering
\fontsize{8}{10}\selectfont
\begin{tabular}{llllllll}
\toprule
\multicolumn{2}{c}{ } & \multicolumn{2}{c}{Class 1} & \multicolumn{2}{c}{Class 2} & \multicolumn{2}{c}{Class 3} \\
\cmidrule(l{3pt}r{3pt}){3-4} \cmidrule(l{3pt}r{3pt}){5-6} \cmidrule(l{3pt}r{3pt}){7-8}
Model Component & Parameter & True & Est. (95\% CrI) & True & Est. (95\% CrI) & True & Est. (95\% CrI)\\
\midrule
\addlinespace[0.3em]
\multicolumn{8}{l}{\textbf{ }}\\
\hspace{1em}MVSN & $\beta_{0}$ & -0.96 & -0.96 (-1.75, -0.34) & -0.21 & -0.27 (-0.62, 0.65) & 0.69 & 0.36 (0.08, 0.62)\\
\hspace{1em}Regression & $\beta_{1}$ & -1.52 & -1.5 (-1.61, -1.39) & -0.09 & -0.15 (-0.27, -0.04) & 1.49 & 1.47 (1.37, 1.58)\\
\hspace{1em} & $\beta_{2}$ & -1.38 & -1.38 (-2.08, -0.79) & 0.52 & 0.68 (0.26, 1.27) & 3.03 & 2.95 (2.66, 3.21)\\
\hspace{1em} & $\beta_{3}$ & -2.37 & -2.33 (-2.44, -2.22) & -0.47 & -0.55 (-0.67, -0.44) & 1.03 & 1.04 (0.93, 1.14)\\
\addlinespace[0.3em]
\multicolumn{8}{l}{\textbf{ }}\\
\hspace{1em} & $\sigma_{11}$ & 1 & 1.03 (0.65, 1.28) & 1 & 0.82 (0.53, 1.13) & 1 & 1.19 (0.99, 1.4)\\
\hspace{1em} & $\sigma_{12}$ & 0.97 & 0.99 (0.62, 1.24) & 0.59 & 0.6 (0.32, 0.85) & -0.98 & -0.9 (-1.06, -0.75)\\
\hspace{1em} & $\sigma_{22}$ & 1 & 1.01 (0.64, 1.25) & 1 & 1.09 (0.76, 1.35) & 1 & 1.17 (0.95, 1.42)\\
\addlinespace[0.3em]
\multicolumn{8}{l}{\textbf{ }}\\
\hspace{1em} & $\psi_{1}$ & -0.33 & -0.27 (-1.06, 0.66) & 0.67 & 0.78 (-0.35, 1.16) & -1 & -0.59 (-0.9, -0.26)\\
\hspace{1em} & $\psi_{2}$ & -0.33 & -0.28 (-1.03, 0.54) & 0.67 & 0.52 (-0.22, 1.02) & -1 & -0.89 (-1.19, -0.52)\\
\addlinespace[0.3em]
\multicolumn{8}{l}{\textbf{ }}\\
\hspace{1em}\hspace{1em}Multinom. & $\delta_{11}$ & 0.45 & 0.39 (0.22, 0.57) & 0.45 & 0.39 (0.22, 0.57) & 0.45 & 0.39 (0.22, 0.57)\\
 & $\delta_{12}$ & -0.04 & 0.01 (-0.18, 0.2) & -0.04 & 0.01 (-0.18, 0.2) & -0.04 & 0.01 (-0.18, 0.2)\\
\addlinespace[0.3em]
\multicolumn{8}{l}{\textbf{ }}\\
\hspace{1em}Clustering & $\pi_l$ & 0.31 & 0.3 (0.28, 0.32) & 0.38 & 0.37 (0.34, 0.39) & 0.32 & 0.33 (0.32, 0.35)\\
\bottomrule
\end{tabular}
\end{table}


\end{document}
