\documentclass[]{article}
\usepackage{lmodern}
\usepackage{amssymb,amsmath}
\usepackage{ifxetex,ifluatex}
\usepackage{fixltx2e} % provides \textsubscript
\ifnum 0\ifxetex 1\fi\ifluatex 1\fi=0 % if pdftex
  \usepackage[T1]{fontenc}
  \usepackage[utf8]{inputenc}
\else % if luatex or xelatex
  \ifxetex
    \usepackage{mathspec}
  \else
    \usepackage{fontspec}
  \fi
  \defaultfontfeatures{Ligatures=TeX,Scale=MatchLowercase}
\fi
% use upquote if available, for straight quotes in verbatim environments
\IfFileExists{upquote.sty}{\usepackage{upquote}}{}
% use microtype if available
\IfFileExists{microtype.sty}{%
\usepackage{microtype}
\UseMicrotypeSet[protrusion]{basicmath} % disable protrusion for tt fonts
}{}
\usepackage[margin=1in]{geometry}
\usepackage{hyperref}
\hypersetup{unicode=true,
            pdftitle={Paper Tables},
            pdfauthor={Carter Allen},
            pdfborder={0 0 0},
            breaklinks=true}
\urlstyle{same}  % don't use monospace font for urls
\usepackage{graphicx,grffile}
\makeatletter
\def\maxwidth{\ifdim\Gin@nat@width>\linewidth\linewidth\else\Gin@nat@width\fi}
\def\maxheight{\ifdim\Gin@nat@height>\textheight\textheight\else\Gin@nat@height\fi}
\makeatother
% Scale images if necessary, so that they will not overflow the page
% margins by default, and it is still possible to overwrite the defaults
% using explicit options in \includegraphics[width, height, ...]{}
\setkeys{Gin}{width=\maxwidth,height=\maxheight,keepaspectratio}
\IfFileExists{parskip.sty}{%
\usepackage{parskip}
}{% else
\setlength{\parindent}{0pt}
\setlength{\parskip}{6pt plus 2pt minus 1pt}
}
\setlength{\emergencystretch}{3em}  % prevent overfull lines
\providecommand{\tightlist}{%
  \setlength{\itemsep}{0pt}\setlength{\parskip}{0pt}}
\setcounter{secnumdepth}{0}
% Redefines (sub)paragraphs to behave more like sections
\ifx\paragraph\undefined\else
\let\oldparagraph\paragraph
\renewcommand{\paragraph}[1]{\oldparagraph{#1}\mbox{}}
\fi
\ifx\subparagraph\undefined\else
\let\oldsubparagraph\subparagraph
\renewcommand{\subparagraph}[1]{\oldsubparagraph{#1}\mbox{}}
\fi

%%% Use protect on footnotes to avoid problems with footnotes in titles
\let\rmarkdownfootnote\footnote%
\def\footnote{\protect\rmarkdownfootnote}

%%% Change title format to be more compact
\usepackage{titling}

% Create subtitle command for use in maketitle
\providecommand{\subtitle}[1]{
  \posttitle{
    \begin{center}\large#1\end{center}
    }
}

\setlength{\droptitle}{-2em}

  \title{Paper Tables}
    \pretitle{\vspace{\droptitle}\centering\huge}
  \posttitle{\par}
    \author{Carter Allen}
    \preauthor{\centering\large\emph}
  \postauthor{\par}
      \predate{\centering\large\emph}
  \postdate{\par}
    \date{3/26/2019}

\usepackage{booktabs}
\usepackage{longtable}
\usepackage{array}
\usepackage{multirow}
\usepackage{wrapfig}
\usepackage{float}
\usepackage{colortbl}
\usepackage{pdflscape}
\usepackage{tabu}
\usepackage{threeparttable}
\usepackage{threeparttablex}
\usepackage[normalem]{ulem}
\usepackage{makecell}
\usepackage{xcolor}

\begin{document}
\maketitle

\begin{tabular}{l|l|l|l|l|l}
\hline
True & Est. (95\textbackslash{}\% CrI) & True & Est. (95\textbackslash{}\% CrI) & True & Est. (95\textbackslash{}\% CrI)\\
\hline
-1.77 & -1.92 (-2.52, -1.38) & 0.11 & 0.24 (-0.16, 1.02) & 1.22 & 0.25 (-0.3, 0.99)\\
\hline
-1.9 & -1.91 (-2.03, -1.8) & -0.01 & 0.06 (-0.15, 0.32) & 1.69 & 1.59 (1.39, 1.83)\\
\hline
-2.43 & -2.48 (-3.11, -1.98) & 0.19 & 0.68 (-0.21, 1.31) & 1.2 & 0.69 (-0.1, 1.24)\\
\hline
-1.94 & -1.9 (-2.04, -1.79) & 0.01 & 0.16 (-0.06, 0.55) & 1.35 & 1.36 (1.16, 1.53)\\
\hline
1 & 1 (0.75, 1.21) & 1 & 0.95 (0.59, 1.81) & 1 & 1.57 (0.9, 1.98)\\
\hline
-0.3 & -0.19 (-0.33, 0.08) & -0.34 & -0.1 (-0.62, 0.3) & -0.78 & -0.45 (-0.76, 0.22)\\
\hline
1 & 1.1 (0.77, 1.39) & 1 & 1.16 (0.81, 1.99) & 1 & 1.09 (0.73, 1.48)\\
\hline
-0.33 & -0.08 (-0.73, 0.69) & 0.67 & 0.51 (-0.41, 0.95) & -1 & 0.32 (-0.59, 0.92)\\
\hline
-0.33 & -0.25 (-0.94, 0.53) & 0.67 & -0.02 (-1, 0.9) & -1 & -0.43 (-1.02, 0.63)\\
\hline
-0.73 & -0.94 (-1.17, -0.72) & -0.73 & -0.94 (-1.17, -0.72) & -0.73 & -0.94 (-1.17, -0.72)\\
\hline
-0.77 & -0.54 (-0.74, -0.36) & -0.77 & -0.54 (-0.74, -0.36) & -0.77 & -0.54 (-0.74, -0.36)\\
\hline
0.33 & 0.43 (0.4, 0.45) & 0.33 & 0.26 (0.22, 0.3) & 0.33 & 0.32 (0.27, 0.35)\\
\hline
\end{tabular}


\end{document}
