\documentclass[]{article}
\usepackage{lmodern}
\usepackage{amssymb,amsmath}
\usepackage{ifxetex,ifluatex}
\usepackage{fixltx2e} % provides \textsubscript
\ifnum 0\ifxetex 1\fi\ifluatex 1\fi=0 % if pdftex
  \usepackage[T1]{fontenc}
  \usepackage[utf8]{inputenc}
\else % if luatex or xelatex
  \ifxetex
    \usepackage{mathspec}
  \else
    \usepackage{fontspec}
  \fi
  \defaultfontfeatures{Ligatures=TeX,Scale=MatchLowercase}
\fi
% use upquote if available, for straight quotes in verbatim environments
\IfFileExists{upquote.sty}{\usepackage{upquote}}{}
% use microtype if available
\IfFileExists{microtype.sty}{%
\usepackage{microtype}
\UseMicrotypeSet[protrusion]{basicmath} % disable protrusion for tt fonts
}{}
\usepackage[margin=1in]{geometry}
\usepackage{hyperref}
\hypersetup{unicode=true,
            pdftitle={Simulation Tables},
            pdfauthor={Carter Allen},
            pdfborder={0 0 0},
            breaklinks=true}
\urlstyle{same}  % don't use monospace font for urls
\usepackage{graphicx,grffile}
\makeatletter
\def\maxwidth{\ifdim\Gin@nat@width>\linewidth\linewidth\else\Gin@nat@width\fi}
\def\maxheight{\ifdim\Gin@nat@height>\textheight\textheight\else\Gin@nat@height\fi}
\makeatother
% Scale images if necessary, so that they will not overflow the page
% margins by default, and it is still possible to overwrite the defaults
% using explicit options in \includegraphics[width, height, ...]{}
\setkeys{Gin}{width=\maxwidth,height=\maxheight,keepaspectratio}
\IfFileExists{parskip.sty}{%
\usepackage{parskip}
}{% else
\setlength{\parindent}{0pt}
\setlength{\parskip}{6pt plus 2pt minus 1pt}
}
\setlength{\emergencystretch}{3em}  % prevent overfull lines
\providecommand{\tightlist}{%
  \setlength{\itemsep}{0pt}\setlength{\parskip}{0pt}}
\setcounter{secnumdepth}{0}
% Redefines (sub)paragraphs to behave more like sections
\ifx\paragraph\undefined\else
\let\oldparagraph\paragraph
\renewcommand{\paragraph}[1]{\oldparagraph{#1}\mbox{}}
\fi
\ifx\subparagraph\undefined\else
\let\oldsubparagraph\subparagraph
\renewcommand{\subparagraph}[1]{\oldsubparagraph{#1}\mbox{}}
\fi

%%% Use protect on footnotes to avoid problems with footnotes in titles
\let\rmarkdownfootnote\footnote%
\def\footnote{\protect\rmarkdownfootnote}

%%% Change title format to be more compact
\usepackage{titling}

% Create subtitle command for use in maketitle
\providecommand{\subtitle}[1]{
  \posttitle{
    \begin{center}\large#1\end{center}
    }
}

\setlength{\droptitle}{-2em}

  \title{Simulation Tables}
    \pretitle{\vspace{\droptitle}\centering\huge}
  \posttitle{\par}
    \author{Carter Allen}
    \preauthor{\centering\large\emph}
  \postauthor{\par}
    \date{}
    \predate{}\postdate{}
  
\usepackage{booktabs}
\usepackage{longtable}
\usepackage{array}
\usepackage{multirow}
\usepackage{wrapfig}
\usepackage{float}
\usepackage{colortbl}
\usepackage{pdflscape}
\usepackage{tabu}
\usepackage{threeparttable}
\usepackage{threeparttablex}
\usepackage[normalem]{ulem}
\usepackage{makecell}
\usepackage{xcolor}

\begin{document}
\maketitle

\begin{table}[t]

\caption{\label{tab:unnamed-chunk-4}Model results for simulated data with n = 1000, k = 4, p = 2, h = 3, r = 2. 1000 iterations were run with a burn in of 250. Missingness mechanism was MAR and P(miss) = 0}
\centering
\fontsize{8}{10}\selectfont
\begin{tabular}{llllllll}
\toprule
\multicolumn{2}{c}{ } & \multicolumn{2}{c}{Class 1} & \multicolumn{2}{c}{Class 2} & \multicolumn{2}{c}{Class 3} \\
\cmidrule(l{3pt}r{3pt}){3-4} \cmidrule(l{3pt}r{3pt}){5-6} \cmidrule(l{3pt}r{3pt}){7-8}
Model Component & Parameter & True & Est. (95\% CrI) & True & Est. (95\% CrI) & True & Est. (95\% CrI)\\
\midrule
\addlinespace[0.3em]
\multicolumn{8}{l}{\textbf{ }}\\
\hspace{1em}MVSN & $\beta_{11}$ & -2.69 & -3.09 (-3.55, -2.45) & 0.42 & 1.06 (0.33, 1.52) & 3.55 & 2.62 (1.91, 3.18)\\
\hspace{1em}Regression & $\beta_{21}$ & -2.97 & -3.14 (-3.24, -3.05) & -0.48 & -0.5 (-0.6, -0.4) & 3.15 & 3.09 (3, 3.19)\\
\hspace{1em} & $\beta_{31}$ & -3.57 & -4.42 (-4.86, -3.51) & -0.07 & 0.6 (-0.36, 1.04) & 3 & 1.99 (1.35, 2.61)\\
\hspace{1em} & $\beta_{41}$ & -3.45 & -3.56 (-3.67, -3.46) & 0.14 & 0.16 (0.05, 0.27) & 3.42 & 3.37 (3.27, 3.47)\\
\hspace{1em} & $\beta_{12}$ & -3.17 & -3.46 (-4.02, -2.86) & 0.29 & 0.91 (0.19, 1.38) & 3.49 & 2.44 (1.79, 3.25)\\
\hspace{1em} & $\beta_{22}$ & -2.46 & -2.64 (-2.73, -2.54) & 0.2 & 0.09 (0, 0.19) & 2.79 & 2.8 (2.7, 2.9)\\
\hspace{1em} & $\beta_{32}$ & -3.58 & -3.97 (-4.47, -3.57) & 0.38 & 1.07 (0.04, 1.5) & 2.84 & 1.87 (1.21, 2.64)\\
\hspace{1em} & $\beta_{42}$ & -3.01 & -3.03 (-3.12, -2.93) & 0.17 & 0.16 (0.06, 0.27) & 3.26 & 3.26 (3.16, 3.35)\\
\addlinespace[0.3em]
\multicolumn{8}{l}{\textbf{ }}\\
\hspace{1em} & $\sigma_{11}$ & 1 & 1.03 (0.78, 1.19) & 1 & 0.96 (0.68, 1.16) & 1 & 1.4 (1.15, 1.6)\\
\hspace{1em} & $\sigma_{12}$ & 0.35 & 0.46 (0.31, 0.62) & 0.69 & 0.73 (0.47, 0.92) & 0.67 & 1.12 (0.87, 1.31)\\
\hspace{1em} & $\sigma_{13}$ & 0.98 & 1.02 (0.76, 1.19) & 0.5 & 0.43 (0.16, 0.6) & -0.17 & 0.38 (0.17, 0.56)\\
\hspace{1em} & $\sigma_{14}$ & 0.32 & 0.39 (0.27, 0.53) & 0.64 & 0.68 (0.41, 0.88) & 0.31 & 0.77 (0.55, 0.95)\\
\hspace{1em} & $\sigma_{22}$ & 1 & 1.25 (0.99, 1.49) & 1 & 1.06 (0.75, 1.33) & 1 & 1.5 (1.21, 1.71)\\
\hspace{1em} & $\sigma_{23}$ & 0.22 & 0.33 (0.18, 0.49) & 0.4 & 0.41 (0.16, 0.6) & 0.13 & 0.73 (0.51, 0.92)\\
\hspace{1em} & $\sigma_{24}$ & 0.26 & 0.51 (0.35, 0.67) & 0.97 & 1.03 (0.71, 1.29) & 0.19 & 0.72 (0.51, 0.91)\\
\hspace{1em} & $\sigma_{33}$ & 1 & 1.04 (0.77, 1.22) & 1 & 0.89 (0.58, 1.09) & 1 & 1.54 (1.22, 1.77)\\
\hspace{1em} & $\sigma_{34}$ & 0.29 & 0.35 (0.22, 0.49) & 0.58 & 0.57 (0.31, 0.78) & 0.83 & 1.31 (0.99, 1.53)\\
\hspace{1em} & $\sigma_{44}$ & 1 & 1.06 (0.9, 1.23) & 1 & 1.06 (0.74, 1.32) & 1 & 1.42 (1.07, 1.64)\\
\addlinespace[0.3em]
\multicolumn{8}{l}{\textbf{ }}\\
\hspace{1em} & $\psi_{1}$ & -0.67 & -0.05 (-0.85, 0.53) & 0.33 & -0.44 (-1, 0.44) & -1.33 & -0.2 (-0.89, 0.7)\\
\hspace{1em} & $\psi_{2}$ & -0.67 & 0.4 (-0.73, 0.94) & 0.33 & -0.53 (-1.04, 0.68) & -1.33 & -0.14 (-0.92, 0.69)\\
\hspace{1em} & $\psi_{3}$ & -0.67 & -0.18 (-0.92, 0.53) & 0.33 & -0.4 (-0.99, 0.48) & -1.33 & -0.01 (-1.02, 0.8)\\
\hspace{1em} & $\psi_{4}$ & -0.67 & -0.19 (-0.68, 0.44) & 0.33 & -0.55 (-1.04, 0.73) & -1.33 & -0.1 (-1.05, 0.74)\\
\addlinespace[0.3em]
\multicolumn{8}{l}{\textbf{ }}\\
\hspace{1em}Multinom. & $\delta_{11}$ & -0.67 & -0.62 (-0.83, -0.42) & -0.67 & -0.62 (-0.83, -0.42) & -0.67 & -0.62 (-0.83, -0.42)\\
\hspace{1em} & $\delta_{12}$ & 0.94 & 0.95 (0.68, 1.23) & 0.94 & 0.95 (0.68, 1.23) & 0.94 & 0.95 (0.68, 1.23)\\
\hspace{1em} & $\delta_{21}$ & 0.34 & 0.34 (0.19, 0.5) & 0.34 & 0.34 (0.19, 0.5) & 0.34 & 0.34 (0.19, 0.5)\\
\hspace{1em} & $\delta_{22}$ & -0.17 & -0.11 (-0.35, 0.12) & -0.17 & -0.11 (-0.35, 0.12) & -0.17 & -0.11 (-0.35, 0.12)\\
\addlinespace[0.3em]
\multicolumn{8}{l}{\textbf{ }}\\
\hspace{1em}Clustering & $\pi_l$ & 0.31 & 0.31 (0.31, 0.32) & 0.28 & 0.27 (0.26, 0.28) & 0.41 & 0.42 (0.41, 0.42)\\
\bottomrule
\end{tabular}
\end{table}


\end{document}
