\documentclass[]{article}
\usepackage{lmodern}
\usepackage{amssymb,amsmath}
\usepackage{ifxetex,ifluatex}
\usepackage{fixltx2e} % provides \textsubscript
\ifnum 0\ifxetex 1\fi\ifluatex 1\fi=0 % if pdftex
  \usepackage[T1]{fontenc}
  \usepackage[utf8]{inputenc}
\else % if luatex or xelatex
  \ifxetex
    \usepackage{mathspec}
  \else
    \usepackage{fontspec}
  \fi
  \defaultfontfeatures{Ligatures=TeX,Scale=MatchLowercase}
\fi
% use upquote if available, for straight quotes in verbatim environments
\IfFileExists{upquote.sty}{\usepackage{upquote}}{}
% use microtype if available
\IfFileExists{microtype.sty}{%
\usepackage{microtype}
\UseMicrotypeSet[protrusion]{basicmath} % disable protrusion for tt fonts
}{}
\usepackage[margin=1in]{geometry}
\usepackage{hyperref}
\hypersetup{unicode=true,
            pdftitle={Simulation Tables},
            pdfauthor={Carter Allen},
            pdfborder={0 0 0},
            breaklinks=true}
\urlstyle{same}  % don't use monospace font for urls
\usepackage{graphicx,grffile}
\makeatletter
\def\maxwidth{\ifdim\Gin@nat@width>\linewidth\linewidth\else\Gin@nat@width\fi}
\def\maxheight{\ifdim\Gin@nat@height>\textheight\textheight\else\Gin@nat@height\fi}
\makeatother
% Scale images if necessary, so that they will not overflow the page
% margins by default, and it is still possible to overwrite the defaults
% using explicit options in \includegraphics[width, height, ...]{}
\setkeys{Gin}{width=\maxwidth,height=\maxheight,keepaspectratio}
\IfFileExists{parskip.sty}{%
\usepackage{parskip}
}{% else
\setlength{\parindent}{0pt}
\setlength{\parskip}{6pt plus 2pt minus 1pt}
}
\setlength{\emergencystretch}{3em}  % prevent overfull lines
\providecommand{\tightlist}{%
  \setlength{\itemsep}{0pt}\setlength{\parskip}{0pt}}
\setcounter{secnumdepth}{0}
% Redefines (sub)paragraphs to behave more like sections
\ifx\paragraph\undefined\else
\let\oldparagraph\paragraph
\renewcommand{\paragraph}[1]{\oldparagraph{#1}\mbox{}}
\fi
\ifx\subparagraph\undefined\else
\let\oldsubparagraph\subparagraph
\renewcommand{\subparagraph}[1]{\oldsubparagraph{#1}\mbox{}}
\fi

%%% Use protect on footnotes to avoid problems with footnotes in titles
\let\rmarkdownfootnote\footnote%
\def\footnote{\protect\rmarkdownfootnote}

%%% Change title format to be more compact
\usepackage{titling}

% Create subtitle command for use in maketitle
\providecommand{\subtitle}[1]{
  \posttitle{
    \begin{center}\large#1\end{center}
    }
}

\setlength{\droptitle}{-2em}

  \title{Simulation Tables}
    \pretitle{\vspace{\droptitle}\centering\huge}
  \posttitle{\par}
    \author{Carter Allen}
    \preauthor{\centering\large\emph}
  \postauthor{\par}
    \date{}
    \predate{}\postdate{}
  
\usepackage{booktabs}
\usepackage{longtable}
\usepackage{array}
\usepackage{multirow}
\usepackage{wrapfig}
\usepackage{float}
\usepackage{colortbl}
\usepackage{pdflscape}
\usepackage{tabu}
\usepackage{threeparttable}
\usepackage{threeparttablex}
\usepackage[normalem]{ulem}
\usepackage{makecell}
\usepackage{xcolor}

\begin{document}
\maketitle

\begin{table}[t]

\caption{\label{tab:unnamed-chunk-5}Model results for simulated data with n = 1000, k = 4, p = 2, h = 3, r = 2. 1000 iterations were run with a burn in of 250. Missingness mechanism was MAR and P(miss) = 0}
\centering
\fontsize{8}{10}\selectfont
\begin{tabular}{llllllll}
\toprule
\multicolumn{2}{c}{ } & \multicolumn{2}{c}{Class 1} & \multicolumn{2}{c}{Class 2} & \multicolumn{2}{c}{Class 3} \\
\cmidrule(l{3pt}r{3pt}){3-4} \cmidrule(l{3pt}r{3pt}){5-6} \cmidrule(l{3pt}r{3pt}){7-8}
Model Component & Parameter & True & Est. (95\% CrI) & True & Est. (95\% CrI) & True & Est. (95\% CrI)\\
\midrule
\addlinespace[0.3em]
\multicolumn{8}{l}{\textbf{ }}\\
\hspace{1em}MVSN & $\beta_{11}$ & 0.03 & -4.11 (-4.33, -3.86) & 2.94 & 9.42 (8.91, 9.77) & 8.98 & -8.37 (-8.86, -0.65)\\
\hspace{1em}Regression & $\beta_{21}$ & 0.15 & -3.98 (-4.09, -3.87) & 3.05 & 11.98 (11.77, 12.18) & 9.52 & -10.85 (-11.1, 0.61)\\
\hspace{1em} & $\beta_{31}$ & 2.34 & -3.74 (-3.99, -3.5) & 2.66 & 11.39 (10.7, 11.78) & 10.88 & -10.29 (-10.78, -0.14)\\
\hspace{1em} & $\beta_{41}$ & 0.18 & -1.97 (-2.08, -1.84) & 3.78 & 14.02 (13.78, 14.22) & 7.88 & -12.89 (-13.13, 0.59)\\
\hspace{1em} & $\beta_{12}$ & 0.05 & 5.83 (5.59, 6.07) & 4.24 & 0.42 (0.03, 0.83) & 8.21 & -0.35 (-0.77, 0.16)\\
\hspace{1em} & $\beta_{22}$ & 1.37 & 4.96 (4.84, 5.07) & 3.06 & 2.02 (1.86, 2.22) & 9.66 & -1.89 (-2.11, 0.02)\\
\hspace{1em} & $\beta_{32}$ & 1.83 & 4.23 (3.96, 4.5) & 3.53 & 0.49 (0.14, 0.86) & 9.36 & -0.37 (-0.77, 0.16)\\
\hspace{1em} & $\beta_{42}$ & 2.97 & 4.96 (4.83, 5.08) & 5.17 & 2.08 (1.92, 2.28) & 7.49 & -1.84 (-2.04, 0.04)\\
\addlinespace[0.3em]
\multicolumn{8}{l}{\textbf{ }}\\
\hspace{1em} & $\Omega_{11}$ & 12.24 & 1.38 (1.2, 1.59) & 71.81 & 3.08 (2.69, 4.16) & 586.47 & 3.49 (2.89, 186.69)\\
\hspace{1em} & $\Omega_{12}$ & 4.13 & 0.2 (0.05, 0.36) & 60.19 & 2.76 (2.39, 3.55) & 573.81 & 3.02 (2.48, 215.36)\\
\hspace{1em} & $\Omega_{13}$ & 5.88 & 0.38 (0.24, 0.54) & 64.04 & 1.9 (1.58, 2.46) & 573.51 & 1.86 (1.42, 8.74)\\
\hspace{1em} & $\Omega_{14}$ & 4.81 & -0.41 (-0.58, -0.26) & 60.87 & 1.78 (1.46, 2.24) & 584.33 & 1.72 (1.29, 12.25)\\
\hspace{1em} & $\Omega_{22}$ & 13.21 & 1.43 (1.22, 1.67) & 76.86 & 3.43 (3.01, 4.23) & 581.74 & 3.71 (3.09, 253.61)\\
\hspace{1em} & $\Omega_{23}$ & 13.96 & 0.05 (-0.1, 0.2) & 73.85 & 2.13 (1.78, 2.7) & 577.64 & 2.02 (1.55, 13.33)\\
\hspace{1em} & $\Omega_{24}$ & 4.01 & 0.64 (0.48, 0.83) & 70.17 & 1.86 (1.53, 2.39) & 575.89 & 1.78 (1.35, 16.8)\\
\hspace{1em} & $\Omega_{33}$ & 17.12 & 1.41 (1.22, 1.64) & 73.47 & 2.55 (2.23, 3.2) & 577.77 & 2.44 (1.99, 5.81)\\
\hspace{1em} & $\Omega_{34}$ & 7.08 & 0.14 (-0.01, 0.3) & 67.84 & 1.99 (1.69, 2.68) & 576.34 & 1.94 (1.55, 5.3)\\
\hspace{1em} & $\Omega_{44}$ & 8.79 & 1.57 (1.37, 1.83) & 76.35 & 2.45 (2.14, 3.79) & 589.19 & 2.37 (1.93, 5.68)\\
\addlinespace[0.3em]
\multicolumn{8}{l}{\textbf{ }}\\
\hspace{1em} & $\alpha_{1}$ & -1.26 & 0 (0, 0) & 4.89 & 0 (0, 0) & -21.89 & 0 (0, 0)\\
\hspace{1em} & $\alpha_{2}$ & -8.28 & 0 (0, 0) & 6.62 & 0 (0, 0) & 1.75 & 0 (0, 0)\\
\hspace{1em} & $\alpha_{3}$ & 8.55 & 0 (0, 0) & -5.87 & 0 (0, 0) & -41.61 & 0 (0, 0)\\
\hspace{1em} & $\alpha_{4}$ & -4.05 & 0 (0, 0) & 0.77 & 0 (0, 0) & 14.01 & 0 (0, 0)\\
\addlinespace[0.3em]
\multicolumn{8}{l}{\textbf{ }}\\
\hspace{1em}Multinom. & $\delta_{11}$ & -0.66 & -0.54 (-0.77, -0.32) & -0.66 & -0.54 (-0.77, -0.32) & -0.66 & -0.54 (-0.77, -0.32)\\
\hspace{1em} & $\delta_{12}$ & -0.26 & -0.26 (-0.6, 0.05) & -0.26 & -0.26 (-0.6, 0.05) & -0.26 & -0.26 (-0.6, 0.05)\\
\hspace{1em} & $\delta_{21}$ & -0.07 & -0.07 (-0.28, 0.14) & -0.07 & -0.07 (-0.28, 0.14) & -0.07 & -0.07 (-0.28, 0.14)\\
\hspace{1em} & $\delta_{22}$ & 0.26 & 0.24 (-0.04, 0.5) & 0.26 & 0.24 (-0.04, 0.5) & 0.26 & 0.24 (-0.04, 0.5)\\
\addlinespace[0.3em]
\multicolumn{8}{l}{\textbf{ }}\\
\hspace{1em}Clustering & $\pi_l$ & 0.39 & 0.39 (0.36, 0.43) & 0.2 & 0.38 (0.13, 0.41) & 0.41 & 0.23 (0.2, 0.44)\\
\bottomrule
\end{tabular}
\end{table}


\end{document}
